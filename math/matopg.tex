
\documentclass{article}
\usepackage[utf8]{inputenc}
\usepackage{amsmath}
\usepackage{amssymb}

\title{MA Funktion 1}
\author{Erik. S. Gimsing}
\date{November 13, 2017}

\begin{document}

\maketitle

\section{Opgave}

Givet er en funktion $f(x) = x^2 + x - 6$. Som definitionsmaengde for $f$ har vi 
alle de reele tal, $\mathbb{R}$. Desuden er vaerdigmaengden saettet 
$\{ y | y\	i\ \mathbb{R}\ og\ -6.25 \leq y\}$, da $f$ har en $y$ vaerdi paa 
$f(-\frac{1}{2})= -6.25$ i sit toppunkt da formlen for $x$ dellen er
$-\frac{b}{2a}$ og den har en positiv $x^2$ koefficient. Denne
funktion skaer $x$-aksen i to punkter:

\begin{align}
	\frac{-b \pm \sqrt{b^2-4ac}}{2a} = \frac{-1 \pm \sqrt{1+4 \cdot 6}}{2} 
	= \frac{-1 \pm 5}{2} = -3,\ 2
\end{align}

Derfor har vi $(-3,0)$ og $(2,0)$. Funktionen skaer desuden i $y$-aksen i
punktet $(0,-6)$ da $f(0) = 0^2 + 0 - 6 = -6 $. Toppunktet er som tidligger
naevnt $(-1/2,-6.25)$. Da vi har positiv koefficient
paa $x^2$ og top punkt $(-1/2,-6.25)$ er $f$ faldenede for $x \leq -1/2$,
konstant for $x = -1/2$ og stigende for $x \geq -1/2$. Paa interval form er
dette $(-\infty, -1/2)$, $[-1/2,-1/2]$ og $(-1/2,+\infty)$. $f$ skaer funktionen
$y = 2x + 5$, disse to skaering er:

\begin{align}
	x^2 + x - 6 &= 2x + 5 \\
	x^2 - x - 11 &= 0 \nonumber \\
	{(x-1/2)}^2 - 11.25 &= 0 \nonumber \\
	x-1/2 &= \pm \sqrt{11.25} \nonumber \\
	x &= 1/2 \pm \sqrt{11.25} \nonumber
\end{align}

Dette giver koordinaterne (afrundet) $(3.85,12.71)$ og $(-2.85,-0.70)$

\section{Opgave}

En funktion er givet ved $f(x) = \frac{3}{x-2} - \frac{1}{2(x-2)}$. $f$ har
definitionsmaengde $\mathbb{R}\backslash \{2\}$, og vaerdigmaengden
$\mathbb{R}\backslash \{0\}$. Nedenfor ses en tegning af $f$
[---------------------------------indset-----------------]

Denne funktion skaer $y$-aksen i punktet $(0,-1.25)$ da $f(0) = \frac{3}{0-2} -
\frac{1}{2(0-2)} = -1.5 + 0.25 = -1.25$. Denne funktion skaer ogsaa med $2x+5$:

\begin{align}
	2x+5 &= \frac{3}{x-2} - \frac{1}{2(x-2)} = \frac{5}{2(x-2)} \\
	2(2x+5)(x-2) &= 5 \nonumber \\
	4x^2+2x-25 &= 0 \nonumber \\
	x &= \frac{-2 \pm \sqrt{404}}{8} \approx 2.26,\ -2.76
\end{align}

Hvilket giver punkterne $(2.26,9.52)$ og $(-2.76,-0.52)$.

\section{Opgave}

Givet er funktionerne:

\begin{itemize}
	\item a. $f(x) = x^2-2$
	\item b. $f(x) = x^6-x^2$
	\item c. $f(x) = x^4-x$
	\item d. $f(x) = \frac{x}{x^2+2}$
\end{itemize}

Det Oenskes at vide om de er lige, ulige eller ingen af delene.

\begin{itemize}
	\item (a) er lige da ${(-x)}^2-2 = x^2-2$
	\item (b) er lige da  ${(-x)}^6-{(-x)}^2 = x^6-x^2$
	\item (c) er ingen af delene da  $ x^4-x \neq {(-x)}^4 +x \neq x - x^4$
	\item (d) er ulige da  $- \frac{x}{x^2+2} = \frac{-x}{{(-x)}^2+2}$
\end{itemize}

\section{Opgave}

Vi skal finde en funktion saaledes at den skaer $(0,-5)$, $(2,1)$ og $(3,7)$. Da
der er 3 punkter vil en andengrads ligning vaere nok. Denne andengrads ligning 
skal have en konstant paa $-5$ da den skal skaer $(0,-5)$. Hvis vi saa nu ligger
5 til de to andre punkters $y$ del faar vi $(2,6)$ og $(3,12)$. Det kan ses at
disse punkter findes i andengrads ligningen $x^2+x$, da $2^2+2 = 6$ og $3^2+3 =
12$. Derfor er den endlige funktion $f(x) = x^2 + x - 5$. Denne funktion skaer
som sagt $y$-aksen i $(0,-5)$ og $x$-aksen i $\frac{-1 \pm \sqrt{21}}{2} \approx
1.79,\ -2.79$ dvs $(1.79,0)$ og $(-2.79,0)$.

\section{Opgave}

Karrets radius maa vaere $h_{total}-h_{bund} = 1.00\ m - 0.20\ m = 0.80\ m$. 

Det antages at $BD$ staar vinkelret paa karret. Dette betyder at $BD$ kan 
forlaenges til at skaere cirklens center. Dette danner en trekant med $y$-aksen
hvis man vaelger at lade cirklen have centrum i $(0\ m,1.00\ m)$. Vi har givet
at der er $1.00$ meter mellem de to ben. Derfor er den anden katets laengde
$0.50$ meter. Saa derfor er vinklen $\arctan(1/0.5) \approx 63.43^o $

Volumenet af karret er:
\begin{align}
\frac{3}{2}\pi{(h_{total}-h_{bund})}^2
= \frac{3}{2}\ m \cdot \pi\cdot 0.80^2\ m^2 \approx 3.02\ m^3
\end{align}

Og overfladearealet er:

\begin{align}
	\pi{(h_{total}-h_{bund})}^2 + 3\pi(h_{total}-h_{bund}) 
	&= \pi \cdot 0.80^2\ m^2 + \pi \cdot 3\ m \cdot 0.8\ m \\
	&\approx 9.55\ m^2 \nonumber
\end{align}

For parable karret er $A = (-0.8, 0.8)$ og $B = (0.8, 0.8)$.
Funktions forskriften til den kan findes ved at se at den maa vaere paa formen
$ax^2$ da den har toppunkt $(0, 0)$. Derfor maa $a$ vaere $\frac{0.8}{0.8^2} =
1.25$, saa derfor er parablen $f(x) = 1.25 \cdot x^2$. Denne funktion har
definitionsmaengden $\mathbb{R}$ og vaerdigmaengden 

$\{y|y\ i\ \mathbb{R}\ og\ 0 \leq y\}$

\end{document}
