\documentclass{article}
\usepackage[utf8]{inputenc}
\usepackage{amssymb}
\usepackage{amsmath}
\usepackage{amsthm}
\usepackage{amsfonts}
\usepackage{tikz}
%\usepackage[margin=0.5in]{geometry}
\theoremstyle{plain}
\newtheorem{theorem}{Theorem}
\newtheorem{claim}[theorem]{Claim}
\newtheorem{proposition}[theorem]{Proposition}
\newtheorem{lemma}[theorem]{Lemma}
\newtheorem{corollary}[theorem]{Corollary}
\newtheorem{conjecture}[theorem]{Conjecture}
\theoremstyle{definition}
\newtheorem*{observation}{Observation}
\newtheorem*{example}{Example}
\newtheorem*{remark}{Remark}
\newtheorem{definition}[theorem]{Definition} 

\title{Movement on graphs}
\author{Erik. S. Gimsing}
\date{January 2018}

\begin{document}
\maketitle
\section{Abstract}
\section{Graphs}
The object of study in this paper is a graph. For this reason a small
write up on the necessary definitions are included. If one is already
familiar it is not imperative to read this section.
\subsection{Definitions}
Two sorts of graphs are considered in this paper. \textit{Graphs} and
\textit{digraphs}. The definition of these two follows.
\begin{definition}[Graph]\label{def graph}
	A graph is an ordered pair $G = (V,E)$ such that
	\begin{itemize}
		\item $V$ is a set, called the \textbf{vertex set}
		\item $E$ is a set of 2-element subset of $V$, called
		the \textbf{edge set}. That is:
			
			$E \subseteq \{\{u,v\}:u,v \in V\}$.
	\end{itemize}
	Then the elements of $V$ are called \textbf{vertices}, and
	the elements of $E$ are called \textbf{edges}. We call $G$
	a \textbf{complete graph} if for all $u,v\in V$ then $\{u,v\}
	\in E$.
\end{definition}
\begin{definition}[Digraph]\label{def digraph}
	A digraph is an ordered pair $D = (V,E)$ such that
	\begin{itemize}
		\item $V$ is a set, called the \textit{vertex set}
		\item $E$ is a set of ordered pair elements of $V^2$, 
			called the \textit{edge set}. That is $E 
			\subseteq V^2$ with the property that $\forall v 
			\in V: (v,v) \notin E$.
	\end{itemize}
	The terminology of a normal graph apply to digraphs, except
	that the elements of $E$ are called \textbf{arcs} instead. A
	digraph $D$ is called a \textbf{symmetric digraph} if $(u,v)
	\in V$ implies that $(v,u) \in V$. We call $D$ a \textbf{simple
	digraph} if $(u,v)\in V$ implies that $(v,u) \notin V$.
\end{definition}
\begin{definition}[Notation]\label{def notation}
	From this point $(a,b)$ will be abreviated to $ab$ if it is a
	move. Further more let
	\begin{equation}
	\begin{aligned}
		\pi_1(p) &= \bigcup \bigcap p \\
		\pi_2(p) &= \bigcup \{x \in \bigcup p: \bigcup p \neq
		\bigcap p \implies x \notin \bigcap p\} \\
	\end{aligned}
	\end{equation}
	That is $\pi_1$ is the first coordinate of $p$ and $\pi_2$
	the second
\end{definition}
\section{Movement on graphs}
\subsection{Definitions}
I will be defining movement on graphs separately. However, as will be
shown, graph movement is the same as movement on a symmetric digraph.
\begin{definition}[Movement on graph]\label{def graph movement}
	Movement on a graph is an ordered pair $\mathcal{M} = (M,+)$,
	where $+$ is a binary operator on $M$, and the properties
	\begin{itemize}
		\item $M \subseteq V^2$ is a \textbf{movement set} with the 
			property that $uv \in M$ iff $\{u,v\} \in E$. Additionally 
			for all $v \in V$, $vv\in M$.
		\item Equality of moves is set equality.
		\item For any $a,b\in M$ let $a+b = \pi_1(a)\pi_2(b)$ iff
			$\pi_2(a) = \pi_1(b)$.
	\end{itemize}
	Then we call $G_M=(V,E,\mathcal{M})$ a graph with movement.
\end{definition}
\begin{definition}[Movement on digraph]\label{def digraph movement}
	Movement on a digraph is an ordered pair $\mathcal{M} = (M,+)$,
	where $+$ is a binary operator on $M$, and the properties
	\begin{itemize}
		\item $M \subseteq V^2$ is a \textbf{movement set} with the 
			property that for all $(u,v) \in E$ we have $uv
			\in M$. Additionally for all $v \in V$, $vv\in M$.
		\item Equality and composition defined as above.
	\end{itemize}
	Then we call $D_M=(V,E,\mathcal{M})$ a digraph with movement.
\end{definition}
\begin{definition}[Applying move to vertices]\label{def apply movement}
	Let $f: M \times V \rightarrow V$ such that 
	\begin{equation}
	\begin{aligned}
		f(m,v) = \pi_2(m) \Leftrightarrow \pi_1(m) = v
	\end{aligned}
	\end{equation}
	This will be abbreviated to $m[v]$ going forward.
\end{definition}
\newpage
\subsection{Results}
\begin{proposition}[Graph movement is 
					symmetric digraph movement]\label{prop1}
	Let $G_M=(V_0,E_0,\mathcal{M}_0)$ be a graph with movement. Then
	there exists a digraph $D_M = (V_1,E_1,\mathcal{M}_1)$ with
	movement such that $D_M$ is symmetric iff $\mathcal{M}_0 =
	\mathcal{M}_1$.
\end{proposition}
\begin{proof}
	\textbf{Necessary condition}

	Let $V_1 = V_0$. Let $(u,v),(v,u) \in E_1$ for all $\{u,v\}
	\in E_0$. Then since $(u,v) \in E_1$ implies $(v,u)\in E_1$,
	then per Definition~\ref{def digraph}, $D_M$ is symmetric.
	Next consider $\mathcal{M}_1 = (M_1,+_1)$. By Definition~\ref{def
	digraph movement} equality and composition is identical. Since
	$V_1 = V_0$ we have that, by Definition~\ref{def graph movement}
	and~\ref{def digraph movement}, that if $vv\in M_0$ then $vv\in
	M_1$. Next let $uv \in M_0$, then $\{u,v\} \in V_0$. Therefore
	by the definition $E_1$ and Definition~\ref{def digraph movement}
	$uv \in M_1$. So $\mathcal{M}_0 = \mathcal{M}_1$
	
	\textbf{Sufficient condition}

	Assume that $\mathcal{M}_0 = \mathcal{M}_1$. Then $uv \in M_0$
	iff $uv \in M_1$. By Definition~\ref{def graph movement} we have
	$\{u,v\} \in E_0$. Since $\{u,v\} = \{v,u\}$ we have $vu \in M_0$
	and thus $vu \in M_1$. So therefore $uv \in M_1$ implies that
	$vu \in M_1$. By Definition~\ref{def digraph} $D_M$ is symmetric.
\end{proof}
\begin{proposition}[Composition is partial mapping]\label{prop2}
	Let $\mathcal{M} = (M,+)$ be a movement structure. Let $x,y \in
	M$, then if for some $z,z' \in M$ we have $x+y=z$ and $x+y=z'$
	then $z=z'$.
\end{proposition}
\begin{proof}
	By Definition~\ref{def graph movement} the equivalence relation on
	$M$ is set equality we have that $=$ is transitive and symmetric
	so therefore $x+y=z$ and $x+y=z'$ implies $z=z'$.
\end{proof}
\begin{proposition}[Composition is assosiative]\label{prop3}
	Let $\mathcal{M} = (M,+)$ be a movement structure. Then for
	all $x,y,z \in M$, $x+(y+z) = (x+y)+z$ if $\pi_2(x)= \pi_1(y)$
	and $\pi_2(y)= \pi_1(z)$, that is if $x+y$ and $y+z$ is defined.
\end{proposition}
\begin{proof}
	For the left side we write
	\begin{equation}
	\begin{aligned}
		x+(y+z) &= x+\pi_1(y)\pi_2(z) && \text{Definition~\ref{def graph
		movement} and $\pi_2(y)= \pi_1(z)$} \\
		&= \pi_1(x)\pi_2(z) && \text{Definition~\ref{def graph
		movement} and $\pi_2(x) = \pi_1(y)$} 
	\end{aligned}
	\end{equation}
	For the right side we write
	\begin{equation}
	\begin{aligned}
		(x+y)+z &= \pi_1(x)\pi_2(y)+z && \text{Definition~\ref{def graph
		movement} and $\pi_2(x)= \pi_1(y)$} \\
		&= \pi_1(x)\pi_2(z) && \text{Definition~\ref{def graph
		movement} and $\pi_2(y)= \pi_1(z)$}
	\end{aligned}
	\end{equation}
	By Definition~\ref{def graph movement} equality of moves is
	set equality.  Therefore $=$ is symmetric and transitive which
	implies that $x+(y+z)=(x+y)+z$
\end{proof}
\begin{proposition}[Composition has left and right 
	per element identity]\label{prop4} Let $\mathcal{M} = (M,+)$
	be a movement structure. Then for all $x \in M$ there exists
	$e_l\in M$ and $e_r\in M$ such that $e_l+x=a$ and $x+e_r = a$.
\end{proposition}
\begin{proof}
	Let $e_l = \pi_1(x)\pi_1(x)$. Then $e_l \in M$ since by
	Definition~\ref{def graph movement} $vv\in M$ for all $v\in
	V$. So we have
	\begin{equation}
	\begin{aligned}
		e_l+x &= \pi_1(x)\pi_2(x) && \text{Definition~\ref{def graph
		movement} and $\pi_2(e_l)= \pi_1(x)$} \\
		&= x && \text{Definition~\ref{def notation}}
	\end{aligned}
	\end{equation}
	Next let $e_r=\pi_2(x)\pi_2(x)$. Like $e_l$, $e_r \in M$. So we have
	\begin{equation}
	\begin{aligned}
		x+e_r &= \pi_1(x)\pi_2(x) && \text{Definition~\ref{def graph
		movement} and $\pi_2(x)= \pi_1(e_r)$} \\
		&= x && \text{Definition~\ref{def notation}}
	\end{aligned}
	\end{equation}
\end{proof}
\begin{proposition}[Graph has inverse element]\label{prop5}
	Let $G_M =(V,E,\mathcal{M})$ be a graph with movement. Then for
	all $x \in M$ there exists $-x \in M$ such that $x+(-x) = e_l$
	and $(-x)+x = e_r$
\end{proposition}
\begin{proof}
	We have that $x \in M$. So by Definition~\ref{def graph movement}
	and~\ref{def notation}, which since $G_M$ is a graph implies that
	$\pi_2(x)\pi_1(x) \in M$ (See Proposition~\ref{prop1}). Let $-x =
	\pi_2(x)\pi_1(x)$, then
	\begin{equation}
	\begin{aligned}
		x+(-x) &= \pi_1(x)\pi_2(-x) && \text{Definition~\ref{def graph
		movement} and $\pi_2(x)= \pi_1(-x)$} \\
		&= \pi_1(x)\pi_1(x) && \text{$\pi_2(-x)= \pi_1(x)$} \\
		&= e_l && \text{Proposition~\ref{prop5}} \\
	\end{aligned}
	\end{equation}
	And for the left side
	\begin{equation}
	\begin{aligned}
		(-x)+x &= \pi_1(-x)\pi_2(x) && \text{Definition~\ref{def graph
		movement} and $\pi_2(-x)= \pi_1(x)$} \\
		&= \pi_2(x)\pi_2(x) && \text{$\pi_2(x)= \pi_1(-x)$} \\
		&= e_r && \text{Proposition~\ref{prop5}} \\
	\end{aligned}
	\end{equation}
\end{proof}
\begin{proposition}[Complete graph has closure]\label{prop6}
	Let $G_M =(V,E,\mathcal{M})$ be a complete graph with
	movement. Then for all $x,y \in M$, $x+y \in M$ if $x+y$
	is defined.
\end{proposition}
\begin{proof}
	By Definition~\ref{def graph movement}, $\{\pi_1(x),\pi_2(x)\}
	\in E$ and $\{\pi_1(y),\pi_2(y)\} \in E$. Therefore by
	Definition~\ref{def graph} we have $\pi_1(x)\in V$ and
	$\pi_2(y) \in V$. By the definition of a complete graph (See
	Definition~\ref{def graph}) we have $\{\pi_1(x),\pi_2(y)\} \in
	E$. By Definition~\ref{def graph movement}, $\pi_1(x)\pi_2(y)
	\in M$. Since $x+y$ is defined $x+y=\pi_1(x)\pi_2(y)$ which
	means that $x+y\in M$.
\end{proof}
\begin{corollary}[Complete graph movement is group-like]\label{cor1}
	Let $G_M=(V,E,\mathcal{M})$ be a complete graph with movement.
	By Proposition~\ref{prop6}, $\mathcal{M}$ has closure. By
	Proposition~\ref{prop3}, $\mathcal{M}$ is associative. By
	Proposition~\ref{prop4}, $\mathcal{M}$ has element wise left
	and right identity. By Proposition~\ref{prop5}, $\mathcal{M}$
	has inverse.
\end{corollary}
\subsection{Applications}
\subsubsection{Graph Theory}
Since movement is defined in terms of graphs, it should be no surprise
that it finds great use in graph theory. Thus far, it is in the form
of nicer definitions of several concepts, phrased in the language of
operators. Below are a few examples.
\begin{proposition}[Equivlance of cycle definition]\label{prop7}
	The digraph $D=(V,E)$ contains a cycle if and only if the movement
	structure of $D$ contains a sequence $a_0,a_1,\dots,a_n\in M$
	of unique moves such that
	\begin{equation}
	\begin{aligned}
		a_0+a_1+\cdots+a_n= e_l
	\end{aligned}
	\end{equation}
	Where $e_l$ is the left identity of $a_0$.
\end{proposition}
\begin{proof}
	\textbf{Necessary condition}: 
	Assume that $D$ contains a cycle. Then there exists a sequence
	$v_0,v_1, \dots,v_n \in V$ of unique vertices such that for $0\leq
	i < n$, $(v_i,v_{i+1}) \in E$ and $(v_n,v_0) \in E$. Let $a_i =
	v_iv_{i+1}$ and $a_n =v_n v_0$ which exists by definition. Then
	we have
	\begin{equation}
	\begin{aligned}
		a_0+a_1+\cdots + a_n &= v_0v_1+v_1v_2+\cdots+v_nv_0 \\
							 &= v_0v_0 = e_l
	\end{aligned}
	\end{equation}
	
	\textbf{Sufficient condition}: 
	Assume there is a sequence $a_0,a_1,\dots,a_n\in M$ of unique
	moves such that $a_0+a_1+\cdots+a_n= e_l$. Let $v_i = \pi_1(a_i)$
	for $1 \leq i \leq n$, note that $v_i$ must be unique otherwise
	$a_i$ would not. Then for $1 \leq j < n$ we have $(v_j,v_{j+1})
	\in E$ by definition and since the sequence equals $e_l$,
	$\pi_2(a_n) = \pi_1(a_0)$ thus $(v_n,v_0)\in E$.
\end{proof}
\begin{remark}
	Note that this definition extends to ordinary graphs by requiring that
	a given vertex only appears in two moves.
\end{remark}
The above proposition is representative of the general use of movement
structures of graphs, that is more compact definitions.
\subsubsection{Aritmethic}
\subsubsection{Computer Science}
\subsubsection{Group Theory}
\subsubsection{Category Theory}
\begin{theorem}[Equivlance of categories and digraphs]
	For every graph, there exists a category isomorphic to the movement
	structure of the graph and for every category there exists a graph with
	isomorphic movement structure.
\end{theorem}
\begin{corollary}
	Every category has an equivalent graph, and vice versa.
\end{corollary}
\section{Conclusion}
\end{document}
