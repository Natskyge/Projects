\documentclass[a4paper,11pt]{article}
\usepackage[scale=0.7,vmarginratio={1:2},heightrounded]{geometry}
\usepackage[utf8]{inputenc}
\usepackage{amssymb}
\usepackage{amsmath}
\usepackage{amsthm}
\usepackage{amsfonts}
\usepackage{tikz}
\usetikzlibrary{graphs,graphs.standard}
%\usepackage[margin=0.5in]{geometry}
\theoremstyle{plain}
\newtheorem{theorem}{Theorem}
\newtheorem{claim}[theorem]{Claim}
\newtheorem{proposition}[theorem]{Proposition}
\newtheorem{lemma}[theorem]{Lemma}
\newtheorem{corollary}[theorem]{Corollary}
\newtheorem{conjecture}[theorem]{Conjecture}
\theoremstyle{definition}
\newtheorem*{observation}{Observation}
\newtheorem*{example}{Example}
\newtheorem*{remark}{Remark}
\newtheorem{definition}[theorem]{Definition} 
\newcommand{\R}{\mathbb{R}}
\newcommand{\Z}{\mathbb{Z}}
\newcommand{\cod}{\text{cod}}
\newcommand{\dom}{\text{dom}}
\newcommand{\Ro}{R_{\circ}}
\newcommand{\iff}{\Longleftrightarrow}

\title{Movement on graphs}
\author{Isol. T. Hontoph}
% Encrypted: Make list of konsonater and vokaler in alphabetical order and do a
% Cesar cipher using them (one to the right).
\date{January 2018}

\begin{document}
\maketitle
\begin{center}
	\begin{tikzpicture}
	  \graph[circular placement, radius=5.5cm,
			 empty nodes, nodes={circle,draw}] {
		\foreach \x in {0,...,5} {
		  \foreach \y in {\x,...,5} {
			\x -- \y;
		  };
		};
		0 --["3"'] 1;
		0 --["2"' near start] 2;
		0 --["1", near start] 4;
		0 --["1",] 5;
		4 --["2"'] 5;
	  };
	  \foreach \x [count=\idx from 0] in {0,...,5} {
		\pgfmathparse{90 + \idx * (360 / 6)}
		\node at (\pgfmathresult:5.9cm) {\x};
	  };
	\end{tikzpicture}
\end{center}
\newpage
\tableofcontents
\newpage
\section{Abstract}
In this paper an algebraic structure is defined based on an underlying
digraph and graph.  It is shown that this algebraic structure has identity
and is associative. Further properties of some special cases of graphs
are considered, where it is shown that symmetric digraphs and graphs have
inverse and complete graphs have have closure. Different applications
are presented with the intention of giving an idea of possible uses for
this algebraic structure.

\section{Graphs}
The object of study in this paper is a graph. For this reason a small
write up on the necessary definitions are included. If one is already
familiar it is not imperative to read this section.
\subsection{Definitions}
Two sorts of graphs are considered in this paper. \textit{Graphs} and
\textit{digraphs}. The definition of these two follows.
\begin{definition}[Graph]\label{def graph}
	A graph is an ordered pair $G = (V,E)$ such that
	\begin{itemize}
		\item $V$ is a set, called the \textbf{vertex set}
		\item $E$ is a set of 2-element subset of $V$, called
		the \textbf{edge set}. That is:
			
			$E \subseteq \{\{u,v\}:u,v \in V\}$.
	\end{itemize}
	Then the elements of $V$ are called \textbf{vertices}, and
	the elements of $E$ are called \textbf{edges}. We call $G$
	a \textbf{complete graph} if for all $u,v\in V$ then $\{u,v\}
	\in E$.
\end{definition}
\begin{definition}[Digraph]\label{def digraph}
	A digraph is an ordered pair $D = (V,E)$ such that
	\begin{itemize}
		\item $V$ is a set, called the \textit{vertex set}
		\item $E$ is a set of ordered pair elements of $V^2$, 
			called the \textit{edge set}. That is $E 
			\subseteq V^2$ with the property that $\forall v 
			\in V: (v,v) \notin E$.
	\end{itemize}
	The terminology of a normal graph apply to digraphs, except
	that the elements of $E$ are called \textbf{arcs} instead. A
	digraph $D$ is called a \textbf{symmetric digraph} if $(u,v)
	\in V$ implies that $(v,u) \in V$. We call $D$ a \textbf{simple
	digraph} if $(u,v)\in V$ implies that $(v,u) \notin V$.
\end{definition}
\begin{definition}[Notation]\label{def notation}
	From this point $(a,b)$ will be abreviated to $ab$ if it is
	a move. Further more for an ordered pair $p$ let $\pi_1(p)$
	denote the first coordinate of $p$ and $\pi_2(p)$ the second.
\end{definition}
\section{Movement on graphs}
\subsection{Definitions}
Movement on graphs and digraphs will be defined separately. However,
as will be shown, graph movement is the same as movement on a symmetric
digraph.
\begin{definition}[Movement on graph]\label{def graph movement}
	Movement on a graph is an ordered pair $\mathcal{M} = (M,+)$,
	where $+$ is a binary operator on $M$, and the properties
	\begin{itemize}
		\item $M \subseteq V^2$ is a \textbf{movement set} with the 
			property that $uv \in M$ iff $\{u,v\} \in E$. Additionally 
			for all $v \in V$, $vv\in M$.
		\item Equality of moves is set equality.
		\item For any $a,b\in M$ let $a+b = \pi_1(a)\pi_2(b)$ iff
			$\pi_2(a) = \pi_1(b)$.
	\end{itemize}
	Then we call $G_M=(V,E,\mathcal{M})$ a graph with movement.
\end{definition}
\begin{definition}[Movement on digraph]\label{def digraph movement}
	Movement on a digraph is an ordered pair $\mathcal{M} = (M,+)$,
	where $+$ is a binary operator on $M$, and the properties
	\begin{itemize}
		\item $M \subseteq V^2$ is a \textbf{movement set} with the 
			property that for all $(u,v) \in E$ we have $uv
			\in M$. Additionally for all $v \in V$, $vv\in M$.
		\item Equality and composition defined as above.
	\end{itemize}
	Then we call $D_M=(V,E,\mathcal{M})$ a digraph with movement.
\end{definition}
\subsection{Results}
\begin{proposition}[Graph movement is 
					symmetric digraph movement]\label{prop1}
	Let $G_M=(V_0,E_0,\mathcal{M}_0)$ be a graph with movement. Then
	there exists a digraph $D_M = (V_1,E_1,\mathcal{M}_1)$ with
	movement such that $D_M$ is symmetric iff $\mathcal{M}_0 =
	\mathcal{M}_1$.
\end{proposition}
\begin{proof}
	\textbf{Necessary condition}

	Let $V_1 = V_0$. Let $(u,v),(v,u) \in E_1$ for all $\{u,v\}
	\in E_0$. Then since $(u,v) \in E_1$ implies $(v,u)\in E_1$,
	then per Definition~\ref{def digraph}, $D_M$ is symmetric.
	Next consider $\mathcal{M}_1 = (M_1,+_1)$. By Definition~\ref{def
	digraph movement} equality and composition is identical. Since
	$V_1 = V_0$ we have that, by Definition~\ref{def graph movement}
	and~\ref{def digraph movement}, that if $vv\in M_0$ then $vv\in
	M_1$. Next let $uv \in M_0$, then $\{u,v\} \in V_0$. Therefore
	by the definition $E_1$ and Definition~\ref{def digraph movement}
	$uv \in M_1$. So $\mathcal{M}_0 = \mathcal{M}_1$
	
	\textbf{Sufficient condition}

	Assume that $\mathcal{M}_0 = \mathcal{M}_1$. Then $uv \in M_0$
	iff $uv \in M_1$. By Definition~\ref{def graph movement} we have
	$\{u,v\} \in E_0$. Since $\{u,v\} = \{v,u\}$ we have $vu \in M_0$
	and thus $vu \in M_1$. So therefore $uv \in M_1$ implies that
	$vu \in M_1$. By Definition~\ref{def digraph} $D_M$ is symmetric.
\end{proof}
\begin{proposition}[Composition is partial mapping]\label{prop2}
	Let $\mathcal{M} = (M,+)$ be a movement structure. Let $x,y \in
	M$, then if for some $z,z' \in M$ we have $x+y=z$ and $x+y=z'$
	then $z=z'$.
\end{proposition}
\begin{proof}
	By Definition~\ref{def graph movement} the equivalence relation on
	$M$ is set equality we have that $=$ is transitive and symmetric
	so therefore $x+y=z$ and $x+y=z'$ implies $z=z'$.
\end{proof}
\begin{proposition}[Composition is assosiative]\label{prop3}
	Let $\mathcal{M} = (M,+)$ be a movement structure. Then for
	all $x,y,z \in M$, $x+(y+z) = (x+y)+z$ if $\pi_2(x)= \pi_1(y)$
	and $\pi_2(y)= \pi_1(z)$, that is if $x+y$ and $y+z$ is defined.
\end{proposition}
\begin{proof}
	For the left side we write
	\begin{equation}
	\begin{aligned}
		x+(y+z) &= x+\pi_1(y)\pi_2(z) && \text{Definition~\ref{def graph
		movement} and $\pi_2(y)= \pi_1(z)$} \\
		&= \pi_1(x)\pi_2(z) && \text{Definition~\ref{def graph
		movement} and $\pi_2(x) = \pi_1(y)$} 
	\end{aligned}
	\end{equation}
	For the right side we write
	\begin{equation}
	\begin{aligned}
		(x+y)+z &= \pi_1(x)\pi_2(y)+z && \text{Definition~\ref{def graph
		movement} and $\pi_2(x)= \pi_1(y)$} \\
		&= \pi_1(x)\pi_2(z) && \text{Definition~\ref{def graph
		movement} and $\pi_2(y)= \pi_1(z)$}
	\end{aligned}
	\end{equation}
	By Definition~\ref{def graph movement} equality of moves is
	set equality.  Therefore $=$ is symmetric and transitive which
	implies that $x+(y+z)=(x+y)+z$
\end{proof}
\begin{proposition}[Composition has left and right 
	per element identity]\label{prop4} Let $\mathcal{M} = (M,+)$
	be a movement structure. Then for all $x \in M$ there exists
	$e_l\in M$ and $e_r\in M$ such that $e_l+x=a$ and $x+e_r = a$.
\end{proposition}
\begin{proof}
	Let $e_l = \pi_1(x)\pi_1(x)$. Then $e_l \in M$ since by
	Definition~\ref{def graph movement} $vv\in M$ for all $v\in
	V$. So we have
	\begin{equation}
	\begin{aligned}
		e_l+x &= \pi_1(x)\pi_2(x) && \text{Definition~\ref{def graph
		movement} and $\pi_2(e_l)= \pi_1(x)$} \\
		&= x && \text{Definition~\ref{def notation}}
	\end{aligned}
	\end{equation}
	Next let $e_r=\pi_2(x)\pi_2(x)$. Like $e_l$, $e_r \in M$. So we have
	\begin{equation}
	\begin{aligned}
		x+e_r &= \pi_1(x)\pi_2(x) && \text{Definition~\ref{def graph
		movement} and $\pi_2(x)= \pi_1(e_r)$} \\
		&= x && \text{Definition~\ref{def notation}}
	\end{aligned}
	\end{equation}
\end{proof}
\begin{proposition}[Graph has inverse element]\label{prop5}
	Let $G_M =(V,E,\mathcal{M})$ be a graph with movement. Then for
	all $x \in M$ there exists $-x \in M$ such that $x+(-x) = e_l$
	and $(-x)+x = e_r$
\end{proposition}
\begin{proof}
	We have that $x \in M$. So by Definition~\ref{def graph movement}
	and~\ref{def notation}, which since $G_M$ is a graph implies that
	$\pi_2(x)\pi_1(x) \in M$ (See Proposition~\ref{prop1}). Let $-x =
	\pi_2(x)\pi_1(x)$, then
	\begin{equation}
	\begin{aligned}
		x+(-x) &= \pi_1(x)\pi_2(-x) && \text{Definition~\ref{def graph
		movement} and $\pi_2(x)= \pi_1(-x)$} \\
		&= \pi_1(x)\pi_1(x) && \text{$\pi_2(-x)= \pi_1(x)$} \\
		&= e_l && \text{Proposition~\ref{prop5}} \\
	\end{aligned}
	\end{equation}
	And for the left side
	\begin{equation}
	\begin{aligned}
		(-x)+x &= \pi_1(-x)\pi_2(x) && \text{Definition~\ref{def graph
		movement} and $\pi_2(-x)= \pi_1(x)$} \\
		&= \pi_2(x)\pi_2(x) && \text{$\pi_2(x)= \pi_1(-x)$} \\
		&= e_r && \text{Proposition~\ref{prop5}} \\
	\end{aligned}
	\end{equation}
\end{proof}
\begin{proposition}[Complete graph has closure]\label{prop6}
	Let $G_M =(V,E,\mathcal{M})$ be a complete graph with
	movement. Then for all $x,y \in M$, $x+y \in M$ if $x+y$
	is defined.
\end{proposition}
\begin{proof}
	By Definition~\ref{def graph movement}, $\{\pi_1(x),\pi_2(x)\}
	\in E$ and $\{\pi_1(y),\pi_2(y)\} \in E$. Therefore by
	Definition~\ref{def graph} we have $\pi_1(x)\in V$ and $\pi_2(y) \in V$. By the definition of a complete graph (See
	Definition~\ref{def graph}) we have $\{\pi_1(x),\pi_2(y)\} \in
	E$. By Definition~\ref{def graph movement}, $\pi_1(x)\pi_2(y)
	\in M$. Since $x+y$ is defined $x+y=\pi_1(x)\pi_2(y)$ which
	means that $x+y\in M$.
\end{proof}
\begin{corollary}[Complete graph movement is group-like]\label{cor1}
	Let $G_M=(V,E,\mathcal{M})$ be a complete graph with movement.
	By Proposition~\ref{prop6}, $\mathcal{M}$ has closure. By
	Proposition~\ref{prop3}, $\mathcal{M}$ is associative. By
	Proposition~\ref{prop4}, $\mathcal{M}$ has element wise left
	and right identity. By Proposition~\ref{prop5}, $\mathcal{M}$
	has inverse.
\end{corollary}
\subsection{Applications}
Before proceeding with the applications, it is worth mentioning one of
the more interesting uses of movement structures, namely as a means for
generating digraphs of algebraic structures. The process is as follows
\begin{definition}[Associated digraph]\label{as digraph}
	Let $(A,*)$ be an algebraic structure with binary operator $*$. Let the
	movement structure have the following property:
	\begin{equation}
	\begin{aligned}
		\forall x,y,z\in A: x*y=z\in A\implies xz,yz\in M
	\end{aligned}
	\end{equation}
	then the graph generated by the movement structure is called the
	\textit{associated digraph}.
\end{definition}
The reason for using movement structures, instead of just defining edges and
vertices's directly is that it turns the question into a question of the
behavior of the movement structure which might be more elegant to work with. For
instance, note that Corollary~\ref{cor1} implies that to show the associated
graph is a complete graph, it is sufficient to show that the movement structure
has closure and inverse. Further more movement structures provide an interesting 
interface with Category Theory.
\subsubsection{Graph Theory}
Since movement is defined in terms of graphs, it should be no surprise
that it finds great use in graph theory. Thus far, it is in the form
of nicer definitions of several concepts, phrased in the language of
operators. Below are a few examples.
\begin{theorem}[Equivlance of cycle definition]\label{the1}
	The digraph $D=(V,E)$ contains a cycle if and only if the movement
	structure of $D$ contains a sequence $a_0,a_1,\dots,a_n\in M$
	of unique moves such that
	\begin{equation}
	\begin{aligned}
		a_0+a_1+\cdots+a_n= e_l
	\end{aligned}
	\end{equation}
	Where $e_l$ is the left identity of $a_0$.
\end{theorem}
\begin{proof}
	\textbf{Necessary condition}: 
	Assume that $D$ contains a cycle. Then there exists a sequence
	$v_0,v_1, \dots,v_n \in V$ of unique vertices such that for $0\leq
	i < n$, $(v_i,v_{i+1}) \in E$ and $(v_n,v_0) \in E$. Let $a_i =
	v_i v_{i+1}$ and $a_n=v_n v_0$ which exists by definition. Then
	we have
	\begin{equation}
	\begin{aligned}
		a_0+a_1+\cdots + a_n &= v_0 v_1+v_1 v_2+\cdots+v_n v_0 \\
							 &= v_0v_0 = e_l
	\end{aligned}
	\end{equation}
	
	\textbf{Sufficient condition}: 
	Assume there is a sequence $a_0,a_1,\dots,a_n\in M$ of unique
	moves such that $a_0+a_1+\cdots+a_n= e_l$. Let $v_i = \pi_1(a_i)$
	for $1 \leq i \leq n$, note that $v_i$ must be unique otherwise
	$a_i$ would not. Then for $1 \leq j < n$ we have $(v_j,v_{j+1})
	\in E$ by definition and since the sequence equals $e_l$,
	$\pi_2(a_n) = \pi_1(a_0)$ thus $(v_n,v_0)\in E$.
\end{proof}
\begin{remark}
	Note that this definition extends to ordinary graphs by requiring
	that a given vertex only appears in two moves.
\end{remark}
The above proposition is representative of the general use of movement
structures of graphs, that is more compact definitions.
\subsubsection{Group Theory}
As was noted in Corollary~\ref{cor1}, movement on a complete graph is
group-like. This leads to an intriguing visual for groups, namely the
following.
\begin{theorem}[Associated digraph of group is 
	complete symmetric digraph]\label{the2}
	Let $(G,*)$ be a group. Then the digraph $D_M=(V,E,\mathcal{M})$,
	where $\mathcal{M}$ is the associated movement structure as
	defined in Definition~\ref{as digraph}, is a complete symmetric
	digraph with movement.
\end{theorem}
\begin{proof}
	As can be seen in Corollary~\ref{cor1}, it is sufficient to show that
	$\mathcal{M}$ has inverse and closure. Note that if we have $ab\in M$, then
	it must be the case that there is some $c\in G$ such that $a*c=b$ or
	$c*a=b$. Then
	\begin{equation}
	\begin{aligned}
		a*c=b &\iff b*c^{-1}=a &&&\text{(Inverse in group)}\\
		c*a=b &\iff c^{-1}*b=a &&&\text{(Inverse in group)}\\
	\end{aligned}
	\end{equation}
	Next, let $v\in M$ and $u\in M$. Note that $v+u=\pi_1(v)\pi_2(u)$. Since a
	group has inverse, there is ${\pi_1(v)}^{-1} \in G$, and by assumption
	$\pi_2(u)\in G$. Let $c={\pi_1(v)}^{-1}*\pi_2(u)$, then
	\begin{equation}
	\begin{aligned}
		\pi_1(v)*c=\pi_2(u)
	\end{aligned}
	\end{equation}
	Therefore, $v+u\in M$, which completes the proof.
\end{proof}
\begin{corollary}[Equivlant graph of associated digraph of group]
	By Theorem~\ref{the2} and Proposition~\ref{prop1} there is a graph with
	identical movement structure for every associated digraph of a group.
\end{corollary}
Using this, we can visualize finite groups more easily. Consider for instance
$\Z/6\Z$, then this is represented as the graph in Figure 1

\begin{figure}[h]\label{fig:ZmodN}
	\centering
	\begin{tikzpicture}
	  \graph[circular placement, radius=1.8cm,
			 empty nodes, nodes={circle,draw}] {
		\foreach \x in {0,...,5} {
		  \foreach \y in {\x,...,5} {
			\x -- \y;
		  };
		};
		0 --["3"'] 1;
		0 --["2"' near start] 2;
		0 --["1", near start] 4;
		0 --["1",] 5;
		4 --["2"'] 5;
	  };
	  \foreach \x [count=\idx from 0] in {0,...,5} {
		\pgfmathparse{90 + \idx * (360 / 6)}
		\node at (\pgfmathresult:2.2cm) {\x};
	  };
	\end{tikzpicture}
	\caption{Graph of the group of integers modulo 6}
\end{figure}

\subsubsection{Category Theory}
The reader might have noticed a few parallels between movement structures
and categories. To elaborate on these, notice that the axioms of
$\mathcal{MOCT}$ (Morphism-only category theory) are the same as what was
shown to hold for movement composition. This becomes even more clear if
we write $\pi_1$ as $\cod$, $\pi_2$ as $\dom$, and $+$ as $\circ$. This
connection would be restricted somewhat, due to Russell's paradox which
would appear with the category of sets. However, further explorations
of these connections are outside the scope of this paper.

\section{Conclusion}
In this paper an algebraic structure $(M,+)$, called a \textit{movement
structure}, is defined by the vertexes of it's underlying digraph or
graph. The composition of \textit{moves} (elements of $M$) is shown
to be associative and have identity. Further more, if the underlying
digraph is symmetric it has inverse and if it is a complete graph it has
closure. This movement structure is used to give an alternative definition
of a cycle in a graph. It is also shown how one might make a graph of
an algebraic structure and that this implies that the graph of a group
is a complete symmetric digraph. Further more it is suggested that the
movement structure might have relations with category theory due to the
similarity of the movement structure and the axioms of $\mathcal{MOCT}$
(Morphism-only category theory).

In conclusion, the movement structure of a graph provide a novel visual
way to look at graphs and various other mathematical objects.
\end{document}
