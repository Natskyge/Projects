\documentclass{article}
\usepackage[utf8]{inputenc}
\usepackage{amssymb}
\usepackage{amsmath}
\usepackage{amsthm}
\usepackage{amsfonts}
%\usepackage[margin=0.5in]{geometry}
\theoremstyle{plain}
\newtheorem{theorem}{Theorem}
\newtheorem{claim}[theorem]{Claim}
\newtheorem{proposition}[theorem]{Proposition}
\newtheorem{lemma}[theorem]{Lemma}
\newtheorem{corollary}[theorem]{Corollary}
\newtheorem{conjecture}[theorem]{Conjecture}
\theoremstyle{definition}
\newtheorem*{observation}{Observation}
\newtheorem*{example}{Example}
\newtheorem*{remark}{Remark}
\newtheorem{definition}[theorem]{Definition} 

\newcommand{\R}{\mathbb{R}}
\newcommand{\N}{\mathbb{N}}
\newcommand{\Z}{\mathbb{Z}}
\newcommand{\Q}{\mathbb{Q}}
\newcommand{\C}{\mathbb{C}}
\newcommand{\dx}{\text{d}}

\title{Calculus}
\author{eriksgimsing }
\date{September 2017}

\begin{document}
\maketitle
\section{Limits}
We adopt the following definition of a \textit{limit}. Let $f: I
\rightarrow \mathbb{R}$ where $I \subseteq \mathbb{R}$ is an open set,
$x,a \in I$ and $\delta,\varepsilon \in \mathbb{R}$
\begin{align*}
    \lim_{x \rightarrow a} f(x) = l &:= \forall \varepsilon > 0 \exists
    \delta > 0:\  0 < |x-a| < \delta \implies |f(x)-l| < \varepsilon \\
\end{align*}  
In order to use this definition more easily we will first prove a few
inequalities.
\begin{theorem}
Let all the following variables be real numbers, then the following is true
\begin{align*}
    \left(|x-y| < \frac{\varepsilon}{2} \land |x_0-y_0| <
    \frac{\varepsilon}{2}\right)
        &\implies |(x+x_0)-(y+y_0)| < \varepsilon \\
    \left(|x-y| < \min \left(1, \frac{\varepsilon}{2(|y_0|+1)} \right)
    \land |x_0-y_0| < \frac{\varepsilon}{2(|y|+1)}\right)
        &\implies |xx_0-yy_0| < \varepsilon \\
    \left(|x-y| < \min \left( \frac{|y|}{2},
    \frac{\varepsilon|y|^2}{2}\right) \land y \neq 0 \right)
        &\implies x \neq 0 \land \left| \frac{1}{x} - \frac{1}{y}\right|
        < \varepsilon
\end{align*}
\end{theorem}
\begin{proof}
\textbf{Part 1:}
\begin{align*}
        |(x+x_0)-(y+y_0)| &= |(x-y)+(x_0-y_0)| \\
                      &\leq |x-y|+|x_0-y_0| \\ &<
                      \frac{\varepsilon}{2}+\frac{\varepsilon}{2}
                      = \varepsilon
\end{align*}

\textbf{Part 2:}
\begin{align*}
    &|x| - |y| \leq |x-y| < 1  \\
    &\implies |x| < |y|+1 
\end{align*}
\begin{align*}
    |xx_0-yy_0| &= |x(x_0-y_0)+y_0(x-y)| \\
                &\leq |x||x_0-y_0| + |y_0||x-y| \\ 
				&< (|y|+1)\frac{\varepsilon}{2(|y|+1)} +
                |y_0|\frac{\varepsilon}{2(|y_0|+1)} \\ 
				&< \frac{\varepsilon}{2} + \frac{\varepsilon}{2} 
				= \varepsilon
\end{align*}
\textbf{Part 3:}
\begin{align*}
    &|y|-|x| \leq |x-y| < \frac{|y|}{2} \\
    &\implies \frac{|y|}{2} < |x| \land x \neq 0 \\
    &\implies \frac{1}{|x|} < \frac{2}{|y|}
\end{align*}
\begin{align}
    \left| \frac{1}{x} - 
	\frac{1}{y}\right| &= \frac{|x-y|}{|xy|} \\
                       &< \frac{2|x-y|}{|y|^2}\\
                       &< \frac{2}{|y|^2} \frac{\varepsilon |y|^2}{2}
					   = \varepsilon
\end{align}
\end{proof}
\begin{theorem}
Let
\begin{align*}
    \lim_{x \rightarrow a} f(x) = l \land \lim_{x \rightarrow a} g(x) = m
\end{align*}
Then
\begin{align*}
	\lim_{x \rightarrow a} f(x) + g(x) &= l + m \\
	\lim_{x \rightarrow a} f(x)g(x) &= lm \\
	l \neq 0 \implies \lim_{x \rightarrow a} \frac{1}{f(x)} &= \frac{1}{l} \\
\end{align*}
\end{theorem}
\begin{proof}
\textbf{Part 1:} We have that for some $\delta_1$ and $\delta_2$ 
\begin{align*}
    0<|x-a|<\delta_1 \implies |f(x)-l| < \varepsilon_1 \\
    0<|x-a|<\delta_2 \implies |g(x)-l| < \varepsilon_2
\end{align*}
Choose $\delta_1$ and $\delta_2$ such that
\begin{align*}
    |f(x)-l| < \frac{\varepsilon}{2} \\
    |g(x)-l| < \frac{\varepsilon}{2}
\end{align*}
And let $\delta = \min(\delta_1, \delta_2)$, then by Theorem 1.1
\begin{align*}
    0 <|x-a| < \delta \implies |(f(x)+g(x))-(l+m)|< \varepsilon
\end{align*}
And therefore
\begin{align*}
    \lim_{x \rightarrow a} f(x) + g(x) = l + m
\end{align*}
\textbf{Part 2:} Like above choose $\delta_1$ and $\delta_2$ such that
\begin{align*}
    |f(x)-l| &< \min \left(1, \frac{\varepsilon}{2(|m|+1)} \right) \\
    |g(x)-m| &< \frac{\varepsilon}{2(|l|+1)}
\end{align*}
And let $\delta = \min(\delta_1, \delta_2)$, then by Theorem 1.1
\begin{align*}
    0 <|x-a| < \delta \implies |f(x)g(x)-lm|< \varepsilon
\end{align*}
Thus
\begin{align*}
    \lim_{x \rightarrow a} f(x) g(x) = l m
\end{align*}
\textbf{Part 3:} By assumption, $l \neq 0$. Choose $\delta$ such that
\begin{align*}
    |f(x)-l| < \min \left( \frac{|y|}{2}, \frac{\varepsilon|l|^2}{2}\right)
\end{align*}
So by Theorem 1.1 $f(x) \neq 0$ and 
\begin{align*}
    0<|x-a|<\delta \implies \left|\frac{1}{f(x)}-\frac{1}{l}\right| < \varepsilon
\end{align*}
\end{proof}
\begin{theorem}
\begin{align*}
    \lim_{x \rightarrow a} f(x) = l \land \lim_{x \rightarrow a} f(x) = m
    \implies l = m
\end{align*}
\end{theorem}
\begin{proof}
Assume $l \neq m$ We have that the limit exists, so let let $\varepsilon
= \frac{|l-m|}{2}$
\begin{align*}
    |l-m| &= |(f(x)-f(x))-(l-m)| \\
          &= |(l-f(x))+(f(x)-m)| \\
          &\leq |f(x)-l|+|f(x)-m|
          < 2\varepsilon
\end{align*}
Which is a contradiction, so our assumption that $l \neq m$ is wrong
and therefore $l = m$.
\end{proof}

\begin{theorem}
	Let $I \subseteq \mathbb{R}$ be an open set. Given functions $f: I
	\rightarrow \mathbb{R}$ and $g: I \rightarrow \mathbb{R}$ if $f(x)=g(x)$ for
	$a \neq x$, then for $a \in I$
	\[\lim_{x \rightarrow a} f(x) = l \implies \lim_{x \rightarrow a} g(x) = l\]
\end{theorem}
\begin{proof}
We have that
\begin{align*}
	\forall \varepsilon > 0 \exists \delta > 0: 0<|x-a|<\delta 
	\implies |f(x)-l| < \varepsilon
\end{align*}
In particular $x \neq a$, since $0<|x-a|$. Thus $f(x)=g(x)$, which means
\begin{align*}
	\lim_{x \rightarrow a} f(x) = \lim_{x \rightarrow a} g(x) = l
\end{align*}

\end{proof}

\section{Continuity}
Let $I \subseteq \mathbb{R}$ be open, $f: I \rightarrow \mathbb{R}$ and 
$a \in I$. Then we say $f$ is continues at $a$ if
\begin{align}
	\lim_{x \rightarrow a} f(x)=f(a)
\end{align}
We say that $f$ is continues on $I$ if the above holds for all $a \in I$.
Further more, $f$ is continues on $[\alpha,\beta]$ if for all 
$\alpha < c < \beta$ $f$ is continues at $c$ and
\begin{align}
	\lim_{x \rightarrow \alpha^-} f(x)=f(\alpha) \land 
	\lim_{x \rightarrow \beta^+} f(x)=f(\beta)
\end{align}

\section{Boundedness}
\subsection{Definition of boundedness}
Let $A \subseteq \mathbb{R}$, we say that $A$ is \textit{bounded above} if there
exists some $b \in \mathbb{R}$ such that for all $a \in A$, $a \leq b$. That is
\begin{align*}
	\exists b \in \mathbb{R}: \forall a \in A: a \leq b
\end{align*}
$A$ is \textit{bounded below} if there is some $c \in \mathbb{R}$ such that $c
\leq a$. That is
\begin{align*}
	\exists c \in \mathbb{R}: \forall a \in A: c \leq a
\end{align*}
\subsection{Infimum and supremum}
Let $A$ be defined as above. We say $A$ has a supremum (also called least upper
bound), written $\sup A$ if there is an upper bound $c$ with the property that 
for all upper bounds $a$, $c\leq a$. Infimum (also called greatest lower bound),
written $\inf A$ is an lower bounded $b$ such that for all lower bounds 
$d$, $d \leq b$.
\subsection{Dedekind completeness of the reals}
$\mathbb{R}$ is dedkind complete, that is for any $A \subseteq \mathbb{R}$ that
is bounded above $\sup A$ exists.

\begin{theorem}
	Let $A \neq \emptyset$ be a bounded set of real numbers. Then $\alpha = \sup
	A$ iff $\alpha$ is an upper bound and
\begin{align*}
	\forall \varepsilon \in \mathbb{R}_{>0}: \exists a \in A: |\alpha-a| <
	\varepsilon
\end{align*}
\end{theorem}
\begin{proof}
	Assume the usual definition, if $\alpha \in A$ then we are done. So assume
	$\alpha \notin A$. Aiming for contradiction, assume that it is not possible 
	for all $\varepsilon$. Then there is $\varepsilon$ such that
	for all $b \in [\alpha-\varepsilon,\alpha]$, $b$ is not in $A$.
	But then $b$ is an upper bound of $A$ and $b \leq \alpha$ meaning that $b$
	is the supremum, contradicting our assumption about $\alpha$.

	Assume next the other definition, again if $\alpha \in A$ we are done. Again
	aiming for contradiction assume $\alpha \neq \sup A$. Then $\sup A <
	\alpha$ and for any $b  \in (\sup A, \alpha] \implies b \notin A$,
	contradicting our assumption about $\alpha$.
\end{proof}

\begin{theorem}
	Suppose $A \neq \emptyset$ is bounded below. Let $-A = \{-x|x\in A\}$. Then
	$-A \neq \emptyset$, $-A$ is bounded above and $-\sup(-A) = \inf A$.
\end{theorem}
\begin{proof}
	Since $A \neq \emptyset$ there is some $a \in A$ so by the definition of
	$-A$ we have $-a \in -A$. Let $c$ be an lower bound of $A$ then for $x \in
	A$ we have
	\begin{align*}
		c \leq x \implies -x \leq -c
	\end{align*}
	So $-A$ is bounded above by $-c$. Since $\mathbb{R}$ is dedekind-complete
	$-A$ has a $\sup(-A)$, that is for all upper bounds $b$
	\begin{align*}
		b \leq \sup(-A) \implies -\sup(-A) \leq -b
	\end{align*}
	But that is the defintion of a largest lower bound so $\inf A = -\sup(-A)$.
\end{proof}
\begin{theorem}
	Let $A \neq \emptyset$ be bounded below and let $B$ a set of lower bounds of
	$A$. Then $B \neq \emptyset$, $B$ is bounded above and $\sup B = \inf
	A$.
\end{theorem}
\begin{proof}
	First note that since $A \neq \emptyset$ is bounded below, and therefore $B
	\neq \emptyset$. Let $a \in A$ and $b \in B$, then by definition
	\begin{align*}
		b \leq a
	\end{align*}
	But that is the definition of an upper bound, so $B$ is bounded above and
	therefore $\sup B$ exists. Thus
	\begin{align*}
		b \leq \sup B \leq a
	\end{align*}
	But that is the definition of $\inf A$, so we have $\sup B = \inf A$.
\end{proof}
\begin{theorem}
	Let $f: [a,b] \rightarrow \mathbb{R}$ such that $f(a) < 0 < f(b)$. Then
	there is $x \in [a,b]$ such that $f(x)=0$ and for all $y \in [a,b]$ such
	that $f(y)=0$ we have $y \leq x$.
\end{theorem}
\begin{proof}
	Let $g(x) = f(-x)$, then $g: [-b,-a] \rightarrow \mathbb{R}$,
	and define that set $A$ as follows
	\begin{align*}
		A = \{x|-b \leq x \leq -a \land \forall y \in [-b,x]: f(y)<0\}
	\end{align*}
	Then by theorem 7-1 in Spivak calculus there is a number $\alpha \in A$ such
	that $g(\alpha)=0$ and for all $\beta \in A$ with $g(\beta) = 0$
	\begin{align*}
		\alpha &\leq \beta \\
		-\beta &\leq -\alpha
	\end{align*}
	But then $-\alpha,-\beta \in [a,b]$ and by the definition of $g$
	\begin{align*}
		g(\alpha) = f(-\alpha) = 0
	\end{align*}
	So $-\alpha$ is the largest number in $[a,b]$ with $f$ equal to zero.
\end{proof}
\begin{theorem}
	If $A = \{x: x < a\}$, then for all $x \in A$ there is $0<\delta_0$ such 
	that for $x_0$ with $|x_0 - x| < \delta_0$, $x_0 \in A$. Further more, for 
	$a < y$ there is $\delta_1$ such that for $x_1$ with $|x_1-y| < \delta_1$,
	$x_1 \notin A$.
\end{theorem}
\begin{proof}
	If $x_0 \leq x < a$ we are done. Therefore, assume $x < x_0$ and let 
	$\delta_0 = a-x$ which is positive since $x < a$, then since $x < x_0$ we
	have $|x_0-x| = x_0-x$, so
	\begin{align*}
		x_0-x<a-x \implies x_0 < a
	\end{align*}
	For the second part, because $a < y$, $y \notin A$. So if $y \leq x_1$ we
	are done. Assume $x_1 < y$, and let $\delta_1 = y-a$, then $|x_1-y| =
	y-x_1$, then
	\begin{align*}
		y-x_1<y-a \implies -x_1 < -a \implies a < x_1
\end{align*}
\end{proof}
\begin{theorem}
	Let $f: [a,b] \rightarrow \mathbb{R}$ be an continues function such that
	$f(a) < 0 < f(b)$, then there is $\alpha \in [a,b]$ such that $f(\alpha) =
	0$
\end{theorem}
\begin{proof}
	Consider the set $B = \{x:a \leq x \leq b \land f(x) < 0\}$. First note that
	$B \neq \emptyset$, since $f(a) < 0$. Next note that $B$ is bounded above by
	$b$. Let $\alpha = \sup B$, aiming for contradiction assume $f(\alpha)
	\neq 0$. First consider $f(\alpha) < 0$, then $\alpha \in B$. Then by the
	above theorem there is $0 < \delta$ such that for $x_0\in B$ if $\alpha 
	< x_0 < \alpha + \delta$ contradicting $\alpha$ being an upper bound.
	Consider next $0 < f(\alpha)$, meaning $\alpha \notin B$. Then by the above 
	theorem there is $\delta$ with $x_1 \notin B$ if $\alpha - \delta < x_1 < 
	\alpha$, but then $0 < x_1 < \alpha$ contradicting $\alpha$ being the least 
	upper bound. So we are forced to conclude that $f(\alpha) = 0$.
\end{proof}
\begin{theorem}
	Let $f: [a,b] \rightarrow \mathbb{R}$ be an continues function such that
	$f(a) < c < f(b)$, then there is $\alpha \in [a,b]$ such that $f(\alpha) =
	0$
\end{theorem}
\begin{proof}
	Let $g: [a,b] \rightarrow \mathbb{R}$ be defined by $g(x) = f(x)-c$. Note
	that $g$ is continues on $[a,b]$ since $f$ is. 
	\begin{align*}
		f(a) <&\ c < f(b) \\
			f(a) - c <&\ 0 < f(b) - c \\
			g(a) <&\ 0 < g(b)
	\end{align*}
	By theorem 3.5, there is $x \in [a,b]$ with $g(x)=0$, thus $f(x)=c$.
\end{proof}
\begin{theorem}
	For all $x,y \in \mathbb{R}$ where $x < y$ there is $q \in \mathbb{Q}$ such
	that $x < q < y$.
\end{theorem}
\begin{proof}
	We have $x < y$, so $0 < y -x$ which means there is $q \in \mathbb{N}$ such
	that 
	\begin{align*}
		\frac{1}{q} < y -x \\
		1 < qy - qx \\
	\end{align*}
	Consider now the interval $[qx,qy]$. This interval is non empty, and by
	the above in inequality, $0 \leq qy-z<1$ implies $z \in [qx,qy]$. Let $S = 
	\{n:n \leq qy\} \subseteq \mathbb{Z}$. By the arcemidian property, $S$ is
	bounded above and therefore admits a $p = \sup S$. Then $p+1 \notin S$,
	meaning $qy < p+1$, so we have $p \leq qy < p+1$, which yields $0 \leq qy-p
	< 1$. Therefore $p\in [qx,qy]$, so we may write
	\begin{align*}
		qx <&\ p < qy \\
		x <&\ \frac{p}{q} < y
	\end{align*}
	The definition of $\mathbb{Q}$ implies that $\frac{p}{q} \in \mathbb{Q}$,
	which completes the proof.
\end{proof}
\begin{theorem}
	Let $r<s$ for $r,s \in \mathbb{Q}$. Then there is $x\in \mathbb{R} \land x
	\notin \mathbb{Q}$ such that $r<x<s$.
\end{theorem}
\begin{proof}
	First note that there is infact an irrational number $i$ between 0 and 1, 
	for example $1/\sqrt{2}$. For any given integers $k\neq n$, their
	difference is at at least 1. So without loss of generality let $k<n$, then
	\begin{align*}
		k < i+k < k+1 \leq n
	\end{align*}
	Let $\frac{a}{c}< \frac{b}{d}$ be rational numbers, then $a < \frac{bc}{d}$ 
	and therefore $ad < bc$. Since $ad,bc \in \mathbb{Z}$ we can write 
	$ad < x < bc$ for irrational $x$. 
	\begin{align*}
		a < \frac{x}{d} < \frac{bc}{d} \\
		\frac{a}{c} < \frac{x}{cd} < \frac{b}{d}
	\end{align*}
	Since $\frac{x}{cd}$ is irrational, this completes the proof.
\end{proof}
\begin{theorem}
	Let $x<y$. Then there is an irrational number between $x$ and $y$.
\end{theorem}
\begin{proof}
	There is $r \in \mathbb{Q}$ such that $x<r<y$. There is also $s\in
	\mathbb{Q}$ such that $x<s<r<y$. Therefore there is irational $i$ between
	$s$ and $r$. So $x<i<y$.
\end{proof}
\subsection{Dense in the reals}
A set $A$ is said to be dense in $\mathbb{R}$ if all $(a,b) \subseteq
\mathbb{R}$ contains element of $A$. That is 
\begin{align*}
	A \text{ is dense in } \mathbb{R} := \forall (x,y) \subseteq \mathbb{R}:
	\exists a \in A: a \in (x,y)
\end{align*}
\begin{theorem}
	Let $f: \mathbb{R} \rightarrow \mathbb{R}$ be a continues function such that
	for all $a$ in dense set $A$ then $f(a)=0$. Then $f$ is 0 for all $x$ in
	$\mathbb{R}$.
\end{theorem}
\begin{proof}
	Aiming for contradiction, assume that $f$ is not 0 for all x. Then for some
	$b$ not in $A$ the value of $f$ is non zero. Then by the continuity of $f$ 
	we must have $\lim_{x \rightarrow b} f(x) = f(b)$. However there is no $0 <
	\delta$ such that for $|x-b| < \delta$ implies $|f(x)-f(b)|<|f(b)|$ since
	$A$ is dense and therefore some $x \in (b-\delta,b+\delta)$ is in $A$,
	meaning that for such $x$ we have $|f(x)-f(b)| = |f(b)| < |f(b)|$.
\end{proof}
\begin{theorem}
	Let $f: \mathbb{R} \rightarrow \mathbb{R}$ and $g: \mathbb{R} \rightarrow 
	\mathbb{R}$ be continues functions such that
	for all $a$ in dense set $A$ then $f(a)=g(a)$. Then $f(x)=g(x)$ for all $x$ 
	in $\mathbb{R}$.
\end{theorem}
\begin{proof}
	Let $h(x) = f(x)-g(x)$, then $h$ is continues and for all $a \in A: h(a) 
	= 0$. By the above theorem, $h(x)=0$ for all $x \in \mathbb{R}$. It follows
	that $f(x)=g(x)$ for all $x$.
\end{proof}
\begin{theorem}
	Let $f: \mathbb{R} \rightarrow \mathbb{R}$ and $g: \mathbb{R} \rightarrow 
	\mathbb{R}$ be continues functions such that
	for all $a$ in dense set $A$ then $f(a) \leq g(a)$. Then $f(x) \leq g(x)$ 
	for all $x$ in $\mathbb{R}$.
\end{theorem}
\begin{proof}
	Assume for contradiction that there exists $b \in \mathbb{R}$ such that
	$f(b)>g(b)$. Let $h(x)=g(x)-f(x)$, then this function is continues, implying
	that
	\begin{align}
		\lim_{x \rightarrow b} h(x) = h(b)
	\end{align}
	However this is not possible, since for every $\delta \in
	\mathbb{R}_{>0}$ there is $a \in [b-\delta,b+\delta]$ such that $a \in A$,
	meaning that for such $a$ we have $0 \leq h(a)$, meaning that for such $b$
	we have $|h(b)-h(a)|$ as a lower bound on $\varepsilon \in \mathbb{R}_{>0}$.
	Thus by contradiction, $f(x) \leq g(x)$.
\end{proof}
\begin{theorem}
	Let $f: \mathbb{R} \rightarrow \mathbb{R}$ be continues and have the
	property $f(x+y) = f(x) + f(y)$ for all $x,y \in \mathbb{R}$. Then there is
	$c \in \mathbb{R}$ such that $f(x)=cx$ for all $x$.
\end{theorem}
\begin{proof}
	First note that $f(x)=f(x)+f(0)\implies f(0)=0$. Next $f(n)+f(-n) = 0
	\implies f(-n) = - f(n)$. Let $c = f(1)$, then assume $f(k) = ck$. We add
	$f(1)$ which yields $f(k+1) = c(k+1)$. Next note that $f(1x) = 1 f(x)$,
	assume $f(mx) = m f(x)$. Then adding $f(x)$ yields $f((m+1)x) = (m+1)f(x)$.
	So let $q \in \mathbb{N}_{>0}$, then $qf(\frac{1}{q}) = c \implies
	f(\frac{1}{q}) = \frac{c}{q}$. Therefore for all $r \in \mathbb{Q}$ we have
	$f(r)=cr$. Then by \textbf{Theorem 3.12} we have $f(x)=cx$.
\end{proof}

\section{Various fundamental results of derivatives}

\textbf{Def:} 

Let $I \subseteq \mathbb{R}$ be an open set and $x \in I$. Then we define the 
\textit{derivative} operator $D$ of a function $f: I \rightarrow \mathbb{R}$ as 
follows:
\begin{align}
    \lim_{h \rightarrow 0} \frac{f(x+h)-f(x)}{h} = D_x f(x)
\end{align}

We say that $f$ is \textit{differentiable} at $x$ if the limit exists. We
will first prove a few equivalences of the derivative which will be useful
in further proofs.

\begin{theorem}
\begin{align}
    \lim_{x \rightarrow a} \frac{f(x)-f(a)}{x-a} = D_a f(a)
\end{align}
\end{theorem}
\begin{proof}
Observe the following
\begin{align}
    \lim_{x \rightarrow a} x = \lim_{h \rightarrow 0} a+h
\end{align}
Therefore
\begin{align}
    \lim_{x \rightarrow a} \frac{f(x)-f(a)}{x-a} &= 
    \lim_{h \rightarrow 0} \frac{f(a+h)-f(a)}{a+h-a} \\ \nonumber
    &= \lim_{h \rightarrow 0} \frac{f(a+h)-f(a)}{h} \\ \nonumber
    &= D_a f(a)
\end{align}
\end{proof}

\begin{theorem}
Let $I \subseteq \mathbb{R}$ be an open set, let $a \in I$, and $f:
I \rightarrow \mathbb{R}$. Then $f$ is differentiable at $a$ if and
only if there is a function $\varphi: I \rightarrow \mathbb{R}$ that is
continuous at $a$ and satisfies:
\begin{align}
    f(x)-f(a)=\varphi(x)(x-a)
\end{align}
Further more $\varphi(a)=D_a f(a)$.
\end{theorem}
\begin{proof}
First assume that $f$ is differentiable at $a$. This means that
\begin{align}
    \lim_{x \rightarrow a} \frac{f(x)-f(a)}{x-a} &=D_a f(a) \\
    \lim_{x \rightarrow a} f(x)-f(a) &= 
    \lim_{x \rightarrow a} D_a f(a) (x-a) \nonumber \\
    \lim_{x \rightarrow a} f(x)-f(a) &= 0 \nonumber \\
    \lim_{x \rightarrow a} f(x) &= f(a) \nonumber
\end{align}

So $f$ is continues at $a$. Let

\begin{align}
    \varphi(x) &= \frac{f(x)-f(a)}{x-a},\ x \neq a \\ \nonumber
    \varphi(a) &= D_a f(a)
\end{align}

Then since $f'(a)$ exists and $f$ is continues at $a$, $\varphi$ is
continues at $a$. Further the more by the definition of $\varphi$ we have

\begin{align}
    f(x)-f(a) = \varphi(x)(x-a)
\end{align}

Which completes the first part. Assume next that there is a
$\varphi$. Then we have

\begin{align}
    f(x)-f(a) &= \varphi(x)(x-a) \\
    \frac{f(x)-f(a)}{x-a} &= \varphi(x) \nonumber
\end{align}

By the continuity of $\varphi$ at $a$ we have that

\begin{align}
    \lim_{x \rightarrow a} \varphi(x) = \lim_{x \rightarrow a}
    \frac{f(x)-f(a)}{x-a} = \varphi(a)
\end{align}

Therefore $\varphi(a) = D_a f(a)$ and thus $f$ is differentiable at $a$
which completes the proof.
\end{proof}

\begin{corollary}
The differentiability of Theorem 1.2 and 1.1 are equivalent. 
\end{corollary}

Using these we may prove a some useful results.

\begin{theorem}
Let $f$ and $g$ be differentiable at $x \in \mathbb{R}$. Then for $\lambda \in \mathbb{R}$ we have:
\begin{enumerate}
	\item $D_x \lambda f(x)= \lambda D_x f(x)$
    \item $D_x(f+g)(x) = D_x f(x)+D_x g(x)$
    \item $D_x (f\cdot g)(x) = (D_x f \cdot g)(x)+(f \cdot D_x g)(x)$
    \item $D_x(f\circ g)(x) = ((D_g f \circ g)\cdot D_x g)(x)$
\end{enumerate}
\end{theorem}
\begin{proof}
\textbf{Part 1:} 
\begin{align}
	f(x_0)-f(x) &= \varphi(x_0)(x_0-x) \\
	\lambda f(x_0)-\lambda f(x) &= \lambda \varphi(x_0)(x_0-x) \nonumber
\end{align}
\textbf{Part 2:}
\begin{align}
	f(x_0)-f(x) &= \varphi(x_0)(x_0-x) \\
	g(x_0)-g(x) &= \gamma(x_0)(x_0-x) \nonumber \\
	(f(x_0)+g(x_0))-(g(x)+f(x))&= (\varphi(x_0)+\gamma(x_0))(x_0-x) \nonumber
\end{align}

\textbf{Part 3:}

\begin{align}
    (f\cdot g)'(x) = \lim_{h \rightarrow 0} \frac{(f\cdot g)(x+h)-(f
    \cdot g)(x)}{h} \\ \nonumber = \lim_{h \rightarrow 0} \frac{(f\cdot
    g)(x+h)-(f \cdot g)(x)+f(x+h)g(x)-f(x+h)g(x)}{h} \\ \nonumber =
    \lim_{h \rightarrow 0} \frac{f(x+h)(g(x+h)-g(x))+g(x)(f(x+h)-f(x))}{h}
    \\ \nonumber = \lim_{h \rightarrow 0} f(x+h)\frac{g(x+h)-g(x)}{h}
    + \lim_{h \rightarrow 0} g(x)\frac{f(x+h)-f(x)}{h} \\ \nonumber =
    (f' \cdot g)(x)+(f \cdot g')(x) \nonumber
\end{align}
\textbf{Part 4:}
\begin{align}
    g(x_0)-g(x)&=\gamma(x_0)(x_0-x) \\ 
    (f\circ g)(x_0) - (f\circ g)(x) &=\varphi(g(x_0))(g(x_0)-g(x)) \\ \nonumber
    &= \varphi(g(x_0))\gamma(x_0)(x_0-x) \\ \nonumber
    \implies &(f \circ g)'(x) = ((f' \circ g)\cdot g')(x) \nonumber
\end{align}
\end{proof}

We may now determine the derivative of various functions.
\section{Integrals}
Before defining integrals we will first define a few other things.

\subsection{Subdivision}
Let $[a,b] \subset \mathbb{R}$. Let $x_i$ for $1 \leq i \leq n$ have the
properties
\begin{align}
	x_0 = a,\ x_n = b,\ x_{i-1} < x_i
\end{align}
Then $\Delta = \{x_i|0 \leq i \leq n\}$ is a \textit{finite subdivision}. If on
the other hand we require only that $0 \leq i$, such that the index set becomes
$\mathbb{N}$ then $\Delta$ is called an \textit{infinite subdivision}. This
subdivision is called \textit{normal} if for any $1 \leq i \leq n$, 
$x_{i-1}-x_i = c$. We also define a function on a finite subdivision 
$\Delta$ called a \textit{norm of subdivision} as follows
\begin{align}
	\|\Delta \| = \max \{x_{i}-x_{i-1}|x_i \in \Delta\}
\end{align}
\subsection{Real sequence}
Let $f: A \rightarrow \R$ be a mapping with $A \subseteq \N$. Choose some
symbol, for instance $a$, then for all $k\in A$ we define $a_k = f(k)$ and we
let ${\langle a_k\rangle}_{k\in A}=f$. Note that $k\in A$ may be replaced with
any condition that holds if and only if $k\in A$.
\subsection{Riemann sum and integral}
Let $f:[a,b]\rightarrow \R$, let $\Delta$ be a subdivision of $[a,b]$
with length $n$ denoting elements of $\Delta$ as $x_i$, and let $C$
be a sequence $\langle c_i \rangle_{1\leq i\leq n}$ such that $c_i \in
[x_{i-1},x_i]$. Let $\Delta x_i = x_i-x_{i-1}$ then the \textit{Riemann sum} is
defined as
\begin{align}
	S(f;\Delta,C)= \sum_{i=1}^n f(c_i)\Delta x_i
\end{align}
Then $f$ is said to be \textit{Riemann integrable} on $[a,b]$ if and only if
\begin{align*}
	\exists L\in\R:\forall\varepsilon\in\R_{>0}:\exists\delta\in\R_{>0}:
	\forall\Delta:\forall C:\|\Delta\|<\delta\implies
	|S(f;\Delta,C)-L|<\varepsilon
\end{align*}
The number $L$ is called the \textit{Riemann integral} of $f$ over $[a,b]$ and
is denoted
\begin{align*}
	\int_a^b f(x)\ \dx x
\end{align*}
\subsection{Properties of integrals}

\section{Results}

\begin{theorem}
\begin{align}
    \frac{d}{dx} \ln x= \frac{1}{x}
\end{align}
\end{theorem}
\begin{proof}
\begin{align}
    \ln x = \int_1^x \frac{dt}{t}
    
\end{align}
\end{proof}


\begin{theorem}
\begin{align}
    \frac{d}{dx} e^x =e^x
\end{align}
\end{theorem}
\begin{proof}
\begin{align}
    \frac{d}{dx} \ln e^x = 1 \\ \nonumber
    \frac{1}{e^x} \frac{d}{dx} e^x = 1 \\ \nonumber
    \frac{d}{dx} e^x =e^x 
\end{align}
\end{proof}


\begin{theorem}
\begin{align}
    \frac{d}{dx} x^a= ax^{a-1}
\end{align}
\end{theorem}
\begin{proof}
\begin{align}
    \frac{d}{dx} x^a &= \frac{d}{dx} e^{a \ln x} \\ \nonumber
                     &= x^a \cdot \frac{a}{x} \\ \nonumber
                     &= ax^{a-1} \nonumber
\end{align}
\end{proof}
\begin{theorem}
\begin{align}
    \frac{d}{dx} \sin x = \cos x \land \frac{d}{dx} \cos x = -\sin x
\end{align}
\end{theorem}
\begin{proof}
\begin{align}
    e^{xi} &= \cos x + i \sin x \\ \nonumber \frac{d}{dx} e^{xi} &=
    \frac{d}{dx} \cos x + \frac{d}{dx} i \sin x \\ \nonumber ie^{xi} &=
    \frac{d}{dx} \cos x + \frac{d}{dx} i \sin x \\ \nonumber i\cos x -
    \sin x &= \frac{d}{dx} \cos x + \frac{d}{dx} i \sin x \\ \nonumber
    \implies \frac{d}{dx} \sin x = \cos x &\land \frac{d}{dx} \cos x =
    -\sin x
\end{align}
\end{proof}

\section{An idea}
Let $\mathcal{F}$ be the set of function on a field $F$ and let $D:
\mathcal{F} \rightarrow \mathcal{F}$ such that for $f,g \in \mathcal{F}$
and $a \in F$:

\begin{align}
    D(f+g)       &= D(f)+D(g) \\
    D(af)        &= aD(f) \\
    D(f\cdot g)  &= gD(f)+fD(g) \\
    D(f \circ g) &= (D(f) \circ g)D(g)
\end{align}

Are any of them redundant and is $D$ $\frac{dy}{dx}$ if $F$ is
$(\mathbb{R},+,\cdot)$
\begin{theorem}
	Let $p_0,p_1,\dots,p_n$ be a list of propositions. Assume
	\begin{align}
		\forall 0 \leq i \leq n: p_i \implies p_k,\ k \equiv i+1 \mod n+1
	\end{align}
	Then for all $p_i$ and $p_j$, $p_i \iff p_j$
\end{theorem}
\begin{proof}
	Let $0 \leq i,j \leq n$, then $i < j$, $i = j$ or $i > j$. We need
	only consider the first since it implies the third and if $i = j$
	then $p_i$ is $p_j$. First note that $p_i$ means that $p_{i+1}$, by
	induction we have $p_i \implies p_j $. Next consider that $p_j$ means that
	if $j < n$ we have $p_{j+1}$. Thus $p_j \implies p_n $. Next consider if $j
	=n$, then $p_n \implies p_0$. By induction $p_0 \implies p_i $. By
	transitivity of implications $p_j \implies p_i$, which completes the proof.
\end{proof}
Does there exists a function $f: \mathbb{R}^2 \rightarrow \mathbb{R}$
such that 
\begin{align}
	\forall \varepsilon \in \mathbb{R}_{>0}
	\exists \delta \in \mathbb{R}_{>0}&: \|x-a\| < \delta \implies
	|f(x)-f(a)|<\varepsilon \\
	\forall y \in \mathbb{R} \exists x \in \mathbb{R}^2&: f(x)=y \\
	\forall x,x' \in \mathbb{R}^2&: x = x' \implies f(x)=f(x')
\end{align}
\newpage
\section{TL:DR}
\subsection{The derivative}
\begin{align*}
	f'(x) &= \lim_{h \rightarrow 0} \frac{f(x+h)-f(x)}{h}\\
	(af(x))' &= af'(x)\\
	(f+g)'(x) &= (f'+g')(x)\\
	(f\cdot g)'(x) &= (f'\cdot g)(x)+(f\cdot g')(x)\\
	(f\circ g)'(x) &= ((f'\circ g)\cdot g')(x)
\end{align*}
\subsection{The integral}
\begin{align*}
	\frac{\dx}{\dx x}\int f(x)\ \dx x &= f(x) \\
	\int_a^b f(x)\ \dx x &= F(x)\big|_a^b=F(b)-F(a)\\
	\int af(x)\ \dx x&= a \int f(x)\ \dx x\\
	\int f(x)+g(x)\ \dx x&= \int f(x)\ \dx x+\int g(x)\ \dx x\\
	\int f(x)g'(x)\ \dx x &= f(x)g(x)-\int f'(x)g(x)\ \dx x\\
	\int f(\varphi(t))\varphi'(t)\ \dx t&=\int f(x)\ \dx x\\
	\int_a^b f(\varphi(t))\varphi'(t)\ \dx t&=\int_{\varphi(a)}^{\varphi(b)} 
	f(x)\ \dx x\\
	\frac{1}{f(x)g(x)} &= \frac{A}{f(x)}+\frac{B}{g(x)}\\
	\frac{\dx}{\dx x}\int_a^b f(x,y)\ \dx y &= \int_a^b \frac{\partial f}
	{\partial x}(x,y)\ \dx y
\end{align*}
\end{document}
