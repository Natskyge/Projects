\documentclass{article}
\usepackage[utf8]{inputenc}
\usepackage{amssymb}
\usepackage{amsmath}
\usepackage{amsthm}
\usepackage{amsfonts}
\usepackage{tikz}
\usepackage{tikz-cd}
%\usepackage[margin=0.5in]{geometry}
\theoremstyle{plain}
\newtheorem{theorem}{Theorem}
\newtheorem{claim}[theorem]{Claim}
\newtheorem{proposition}[theorem]{Proposition}
\newtheorem{lemma}[theorem]{Lemma}
\newtheorem{corollary}[theorem]{Corollary}
\newtheorem{conjecture}[theorem]{Conjecture}
\theoremstyle{definition}
\newtheorem*{observation}{Observation}
\newtheorem*{example}{Example}
\newtheorem*{remark}{Remark}
\newtheorem{definition}[theorem]{Definition} 

\newcommand{\R}{\mathbb{R}}
\newcommand{\N}{\mathbb{N}}
\newcommand{\Z}{\mathbb{Z}}
\newcommand{\Q}{\mathbb{Q}}
\newcommand{\C}{\mathbb{C}}
\newcommand{\dx}{\text{d}}
\newcommand{\Sin}{\text{Sin}}
\newcommand{\del}{\partial}


\title{Topology and Homology}
\author{Erik. S. Gimsing}
\date{February 2018}

\begin{document}
\maketitle
\section{Continuity}
We wish to arrive at a general definition of continuity that does not appeal to
the real numbers. To start with we have.
\begin{definition}[Continuity I]
	Let $f:I\rightarrow\R$ be a real function, where $I\subseteq\R$. Let
	$a\in I$. Then $f$ is continues at $a$ iff
	\begin{equation}
	\begin{aligned}
		\forall\varepsilon\in\R_{>0}:\exists\delta\in\R_{>0}:|x-a|<\delta
		\implies|f(x)-f(a)|<\varepsilon
	\end{aligned}
	\end{equation}
\end{definition}
\begin{remark}
The above definition has a natural extension if we consider the definition of
Euclidean distance.
\end{remark}
\begin{definition}[Euclidean distance]
	Let $x=(x_1,\dots,x_n)\in\R^n$ and $y=(y_1,\dots,x_n)\in\R^n$. Then the
	euclidean distance between $x$ and $y$ is given by
	\begin{equation}
	\begin{aligned}
		d(x,y)=\|x-y\|=\sqrt{{(x_1-y_1)}^2+\cdots+{(x_n-y_n)}^2}
	\end{aligned}
	\end{equation}
\end{definition}
\begin{remark}
	We note that in the case of $\R^1$ the above definition reduces to
	$\sqrt{{(x-y)}^2}=|x-y|$. This observation leads us to the following
	definition.
\end{remark}
\begin{definition}[Continuity II]
	Let $f:I\rightarrow\R^m$ be a real function, where $I\subseteq\R^n$. Let
	$a\in I$. Then $f$ is continues at $a$ iff
	\begin{equation}
	\begin{aligned}
		\forall\varepsilon\in\R_{>0}:\exists\delta\in\R_{>0}:\|x-a\|<\delta
		\implies\|f(x)-f(a)\|<\varepsilon
	\end{aligned}
	\end{equation}
\end{definition}
\begin{remark}
	Note that Continuity II reduces to Continuity I when $m=n=1$ due to
	$\|x-a\|=|x-a|$ and like wise with $\|f(x)-f(a)\|=|f(x)-f(a)|$. Therefore
	Continuity II is an extension of Continuity I.
\end{remark}
Euclidean distance derives from Pythoagoras' Theorem which is about triangles.
Triangles are closely related to spheres in all dimensions. Motivated by this we
will seek to define continuity in terms of spheres. At the same time we will try
to free ourselves from euclidean space using the following definition.
\begin{definition}[Metric space]\label{def:metric space}
	A metric space $M=(A,d)$ is an ordered pair consisting of\newline
	$(1):\quad \text{a non-empty set }A$\newline
	together with:\newline
		$(2):\quad \text{a real-valued function } d:A^2\rightarrow\R \text{
		satisfying:}$
	\begin{equation}
	\begin{aligned}
		(M1)\; &:\forall x\in A &&:d(x,x)=0&&\\
		(M2)\; &:\forall x,y,z\in A &&:d(x,y)+d(y,z)\leq d(x,z)&&\\
		(M3)\; &:\forall x,y\in A &&:d(x,y)=d(y,x)&&\\
		(M4)\; &:\forall x,y\in A &&: x\neq y\implies d(x,y)>0&&\\
	\end{aligned}
	\end{equation}
\end{definition}
\begin{remark}
	The ordered pair $(\R^n,\|\cdot\|)$ satisfies
	Definition~\ref{def:metric space}, and thus euclidean space is
	a metric space. This is not a coincidence, since the definition
	of a metric space is based on how euclidean space behaves.
\end{remark}
\begin{definition}[Open-ball]\label{def:open-ball}
	Let $M=(A,d)$ be a metric space. Let $a\in A$ and let
	$\varepsilon\in\R_{>0}$. The \textbf{open $\varepsilon$-ball of $a$ in $M$}
	is defined as
	\begin{equation}
	\begin{aligned}
		B_{\varepsilon}(a):=\left\{x\in A: d(x,a)<\varepsilon\right\}
	\end{aligned}
	\end{equation}
\end{definition}
\begin{remark}
	In the case of $M=(\R^n,\|\cdot\|)$ the above is the same as the inside of a
	sphere with radius $\varepsilon$ and center $a$ without the border. The
	missing border will turn out to be the most important property, but first we
	will use it the extend the definition of continuity.
\end{remark}
\begin{definition}[Continuity III]
	Let $(A,d_1)$ and $(B,d_2)$ be metric spaces. Let $f:I\rightarrow B$ where
	$I\subseteq A$. Let $a\in I$. Then $f$ is continues at $a$ iff
	\begin{equation}
	\begin{aligned}
		\forall\varepsilon\in\R_{>0}:\exists\delta\in\R_{>0}:
		f(B_{\delta}(a))\subseteq B_{\varepsilon}(f(a))
	\end{aligned}
	\end{equation}
\end{definition}
\begin{theorem}[Continuity III is extension of Continuity II]
	A real function $f:I\rightarrow\R^m$, where $I\subseteq\R^n$
	is Continuity II continues at $a\in I$ iff it's Continuity III
	continues at $a$.
\end{theorem}
\begin{proof}$\ $\newline
	\textbf{Necessary Condition}\newline
	Suppose $f$ is Continuity II continues at $a\in
	I\subseteq\R^n$. Let $\varepsilon\in\R_{>0}$ be given.
	Then there is $\delta\in\R_{>0}$ such that $\|x-a\|<\delta$
	implies $\|f(x)-f(a)\|<\varepsilon$. Consider now the open-ball
	with radius $\delta$ and center $a$.
	\begin{equation}
	\begin{aligned}
		B_{\delta}(a)=\{x\in I:\|x-a\|<\delta\}
	\end{aligned}
	\end{equation}
	Then for all $x\in B_{\delta}(a)$ we have $\|x-a\|<\delta$, which implies
	that $\|f(x)-f(a)\|<\varepsilon$ and thus $f(x)\in B_{\varepsilon}(f(a))$.
	Since $x$ was arbitrary, $f(B_{\delta}(a))\subseteq B_{\varepsilon}(f(a))$
	meaning that $f$ is Continuity III continues at $a$.
	\newline\newline
\textbf{Sufficient Condition}\newline
	Suppose $f$ is Continuity III continues at $a\in
	I\subseteq\R^n$. Let $\varepsilon\in\R_{>0}$ be given.	Then there
	is $\delta\in\R_{>0}$ such that $f(B_{\delta}(a))\subseteq
	B_{\varepsilon}(f(a))$. Let $x\in B_{\delta}(a)$ be arbitrary. Then
	$\|x-a\|<\delta$ and $f(x)\in B_{\varepsilon}(f(a))$, that is
	$\|f(x)-f(a)\|<\varepsilon$. Therefore $\|x-a\|<\delta$ implies
	$\|f(x)-f(a)\|$ meaning that $f$ is Continuity II continues at $a$.
\end{proof}
\begin{remark}
	The reason the above is useful is because among the subsets of $\R^n$ are
	sets with nice properties called \textbf{open} and \textbf{closed} sets.
\end{remark}
\begin{definition}[Open Set I]
	Let $M=(A,d)$ be a metric space. Let $U\subseteq A$. Then $U$ is an
	\textbf{open set in $M$} if and only if for all $y\in U$ there is
	$\varepsilon\in\R_{>0}$ such that $B_{\varepsilon}(y)\subseteq U$. That is
	\begin{equation}
	\begin{aligned}
		\forall y\in U:\exists\varepsilon\in\R_{>0}: 
		B_{\varepsilon}(y)\subseteq U
	\end{aligned}
	\end{equation}
\end{definition}
\begin{definition}[Closed Set I]
	Let $M=(A,d)$ be a metric space. Let $H\subseteq A$. Then $H$ is an
	\textbf{closed set in $M$} if and only if its complement $A\backslash H$ is
	open in $M$
\end{definition}
Using this we may define continuity in terms of open sets.
\begin{definition}[Continuity IV]
	Let $M_1=(A_1,d_1)$ and $M_2=(A_2,d_2)$ be metric spaces. Let
	$f:A_1\rightarrow A_2$. Then $f$ is continues at all $x\in A_1$
	if and only if $f^{-1}(U)$ is an open set in $M$ when $U$ is an
	open set in $N$.
\end{definition}
\begin{theorem}[Continuity VI iff Continuity III]
	Let $M_1=(A_1,d_1)$ and $M_2=(A_2,d_2)$ be metric spaces. Let
	$f:A_1\rightarrow A_2$. Then $f$ is Continuity III continues iff its
	Continuity IV continues.
\end{theorem}
\begin{proof}$\ $\newline
	\textbf{Necessary Condition}\newline
	Suppose $f$ is Continuity III continues. Let $U$ be an open set in $M_2$.
	Let $x\in f^{-1}(U)$. Since $f(x)\in U$ which is open there is
	$\varepsilon\in\R_{>0}$ such that $B_{\varepsilon}(f(x))\subseteq U$. Since
	$f$ is Continuity III continues there is $\delta\in\R_{>0}$ such that
	$f(B_{\delta}(x))\subseteq B_{\varepsilon}(f(x))$. Therefore
	\begin{equation}
	\begin{aligned}
		B_{\delta}(x)\subseteq f^{-1}(f(B_{\delta}(x))) \subseteq 
		f^{-1}(B_{\varepsilon}(f(x)))\subseteq f^{-1}(U)
	\end{aligned}
	\end{equation}
	$x$ was arbitrary and therefore $f^{-1}(U)$ is open in $M_1$. Therefore $f$
	is Continuity IV continues.
	\newline\newline
	\textbf{Sufficient Condition}\newline
	Suppose now that $f$ is Continuity IV continues. Let $x\in A_1$, and let
	$\varepsilon\in\R_{>0}$. Then $B_{\varepsilon}(f(x))$ is an open set in
	$M_2$. Therefore since $f$ is Continuity IV continues,
	$f^{-1}(B_{\varepsilon}(f(x)))$ is an open set in $M_1$. Notice that
	$x\in f^{-1}(B_{\varepsilon}(f(x)))$. Therefore since
	$f^{-1}(B_{\varepsilon}(f(x)))$ is an open set in $M_1$ there is
	$\delta\in\R_{>0}$ such that $B_{\delta}(x)\subseteq
	f^{-1}(B_{\varepsilon}(f(x)))$. Then for all $x_0\in B_{\delta}(x)$ there is
	$y_0\in B_{\varepsilon}(f(x))$ such that $f(x_0)=y_0$. Therefore
	$f(B_{\delta}(x))\subseteq B_{\varepsilon}(f(x))$, so $f$ is Continuity III
	continues.
\end{proof}
The only place where we still use the real numbers in Continuity IV is with the
definition of an open set. But it turns out there is a suitably generalization
of what it means for at set to be open. This leads to the following.
\begin{theorem}[Open set properties]\label{Openball is openset}
	Let $M=(A,d)$ be a metric space. Then the open sets in $M$
	satisfy the following properties. 
	\begin{equation}
	\begin{aligned}
		&(O1)\; :\quad \text{The arbitrary union of open sets is open}\\
		&(O2)\; :\quad \text{If $U$ and $O$ are open, then so is $U\cap O$}\\
		&(O3)\; :\quad \text{$A$ is an open set}\\
	\end{aligned}
	\end{equation}
\end{theorem}
\begin{proof}$\ $\newline
\begin{itemize}
	\item[(\textit{O}1)] Let $I$ be an index set and let $i\in I$ and $A_i$ be
		an open set in $M$. Let $x\in \bigcup_{i\in I} A_i$. Then there must be
		$i'\in I$ such that $x\in A_{i'}$, and since $A_{i'}$ is open in $M$
		there is $\varepsilon\in\R_{>0}$ such that $B_{\varepsilon}(x)\subseteq
		A_{i'}\subseteq \bigcup_{i\in I} A_i$. Since $x$ was arbitrary, the
		arbitary union must be open.
	\item[(\textit{O}2)] let $x\in U\cap O$, since $U$ and $O$ are open there
		are $\varepsilon_1\in\R_{>0}$ and $\varepsilon_2\in\R_{>0}$ such that 
		$B_{\varepsilon_1}(x)\subseteq U$ and $B_{\varepsilon_2}(x)\subseteq O$.
		Let $\varepsilon = \min(\varepsilon_1,\varepsilon_2)$, then if
		$\varepsilon_1\leq\varepsilon_2$ we have
	\begin{equation}
	\begin{aligned}
		B_{\varepsilon}(x)= B_{\varepsilon_1}(x)
		\subseteq B_{\varepsilon_2}(x)\subseteq O
	\end{aligned}
	\end{equation}
		therefore
	\begin{equation}
	\begin{aligned}
		B_{\varepsilon}(x) = B_{\varepsilon_1}(x) \cap B_{\varepsilon}(x)
		\subseteq B_{\varepsilon_1}(x) \cap B_{\varepsilon_2}(x)
		\subseteq U \cap O 
	\end{aligned}
	\end{equation}
		Since $x$ was arbitrary, $U\cap O$ is open.
	\item[(\textit{O}3)] By definition any open-ball in $M$ is a subset of $A$.
		Hence the result.
\end{itemize}
\end{proof}
\begin{remark}
	It turns out that the above properties are all we care about for open sets,
	since we may always reconstruct our original concept of an open set. The
	advantage is that the above properties do not appeal to real numbers in any
	way. Hence the following definition.
\end{remark}
\begin{definition}[Topological Space]
	Let $S$ be a set. Let $\tau \subseteq \mathcal{P}(S)$, called a topology on
	$S$, satisfy the
	\textbf{open set axioms}:
	\begin{equation}
	\begin{aligned}
		&(O1)\; :\quad \text{The union of an abitary subset of $\tau$ is an
				 element of $\tau$.}\\
		&(O2)\; :\quad \text{The intersection of any two elements of $\tau$ is
				 an element of $\tau$.}\\
		&(O3)\; :\quad \text{$S$ is an element of $\tau$.}\\
	\end{aligned}
	\end{equation}
	Then the ordered pair $(S,\tau)$ is called a \textbf{topological space} and
	the elements of $\tau$ are called open sets of $(S,\tau)$.
\end{definition}
Using this we arrive at our final definition of continuity.
\begin{definition}[Continuity V]
	Let $(S_1,\tau_1)$ and $(S_2,\tau_2)$ be topological spaces. Let
	$f:S_1\rightarrow S_2$ be a mapping. Then $f$ is continues on $S_1$ if and
	only if
	\begin{equation}
	\begin{aligned}
		U\in\tau_2 \implies f^{-1}(U)\in\tau_1
	\end{aligned}
	\end{equation}
\end{definition}
\begin{remark}
	Note that Continuity IV is a special case of Continuity V
	where the topology on a metric space $(A,d)$ is defined as
	$\tau=\{B_{\varepsilon}(x):x\in\}$ and $\tau$ clearly satisfies
	the open set axioms as was shown in Theorem~\ref{Openball is
	openset}. So Continuity V is an extension of Continuity IV.
\end{remark}
When we extended the definition continuity with Continuity III we stopped
considering point-wise continuity. However, we may now define point-wise
continuity in general using our new definition of open sets.
\begin{definition}[Neighborhood]
	Let $(S,\tau)$ be a topological space. Let $z\in S$ be a point of $S$. Then
	a neighborhood is a set $N_z\subseteq S$ such that there is a set $U\in
	\tau$, such that $U\subseteq N_z$ and $z\in U$. That is
	\begin{equation}
	\begin{aligned}
		\exists U\in\tau: z\in U\subseteq N_z \subseteq S
	\end{aligned}
	\end{equation}
	We call $N_z$ an open neighborhood if $N_z\in\tau$.
\end{definition}
\begin{definition}[Continuity V, point-wise]
	Let $T_1=(S_1,\tau_1)$ and $T_2=(S_2,\tau_2)$ be topological spaces. Let
	$f:S_1\rightarrow S_2$ be a mapping. Then $f$ is continues at $x\in S_1$ 
	if and only if for every neighborhood of $f(x)$ in $T_2$, there exists a
	neighborhood $M$ of $x$ in $T_1$ such that $f(M)\subseteq N$.
\end{definition}
\begin{theorem}[Equivalence of Continuity V Definitions]
	Let $T_1=(S_1,\tau_1)$ and $T_2=(S_2,\tau_2)$ be topological spaces. Let
	$f:S_1\rightarrow S_2$ be a mapping. Then $f$ is continues on $S_1$ if and
	only if it is point-wise continues for all $x\in S_1$.
\end{theorem}
\begin{proof}$\ $\newline
	\textbf{Necessary Condition}\newline
	Suppose that $f$ is continues for all points in $S_1$. Let
	$U\in \tau_2$, we wish to show that $f^{-1}(U)$. If
	$f^{-1}(U)=\emptyset$ we are done since $\emptyset\subseteq\tau_1$
	(because $\emptyset$ is a subset of all sets) and by the
	definition of a topological space $\emptyset=\bigcup \emptyset
	\in \tau_1$. Assume therefore that $f^{-1}(U)\neq\emptyset$
	and let $x\in f^{-1}(U)$. Then for all neighborhoods $N$ of $x$ in $T_1$
	we have $f(N)\subseteq U$. By the definition of a neighborhood, there is
	$X\in\tau_1$ such that $X\subseteq N$. Therefore $f(X)\subseteq
	f(N)\subseteq U$. Let $\mathcal{C}=\{X\in\tau_1:f(X)\subseteq U\}$, and let
	$H=\bigcup\mathcal{C}$. Then by the definition of a topological space,
	$H\in\tau_1$ and clearly $f^{-1}(U)\subseteq H$. But $H\subseteq f^{-1}(U)$,
	therefore $f^{-1}(U)=H$ meaning that $f^{-1}(U)\in\tau_1$.
	\newline\newline
	\textbf{Sufficient Condition}\newline
	Suppose that
	\begin{equation}
	\begin{aligned}
		U\in \tau_2 \implies f^{-1}(U)\in \tau_1
	\end{aligned}
	\end{equation}
	Let $x\in S_1$ and let $N\subseteq S_2$ be a neighborhood of
	$f(x)$ in $T_2$. By definition there is $U\in\tau_2$ such
	that $U\subseteq N$. Then $f^{-1}(U)\in\tau_1$, which is a
	neighborhood, and $f(f^{-1}(U))\subseteq U\subseteq N$ as desired.
\end{proof}
Our last major definition is one of the most important ones in topology,
it is a way to tell if two topological spaces are equivalent.
\begin{definition}[Homeomorphism]\label{homeomorphism}
	Let $T_1=(S_1,\tau_1)$ and $T_2=(S_2,\tau_2)$ be topological spaces. Let
	$f:S_1\rightarrow S_2$ be a continues mapping on $S_1$. We call $f$ a
	\textit{homeomorphism} if there exists a continues mapping $g:S_2\rightarrow
	S_1$ on $S_2$ such that
	\begin{equation}
	\begin{aligned}
		g(f(x))=x \text{ and } f(g(y))=y
	\end{aligned}
	\end{equation}
	for all $x\in S_1$ and $y\in S_2$. Then we call $T_1$ and $T_2$
	\textit{homeomorphic}, written as $T_1 \simeq T_2$.
\end{definition}
It turns out that homeomorphism is actually an equivlance relation, just like
ordinary equality.
\begin{theorem}[Homeomorphism Relation is Equivalence]
	Let $T_i$ for $i\in\N$ denote a topological space. Then the following
	properties hold
	\begin{equation}
	\begin{aligned}
		(1)\; &:\quad\forall T_1 &&: T_1\simeq T_1 &&&\text{(Reflexive)}\\
		(2)\; &:\quad\forall T_1,T_2 &&: T_1\simeq T_2\implies T_2\simeq T_1 
		&&&\text{(Symmetric)}\\
		(3)\; &:\quad\forall T_1,T_2,T_3 &&: T_1\simeq T_2\land T_2\simeq T_3
		\implies T_1\simeq T_3 &&&\text{(transitive)}\\
	\end{aligned}
	\end{equation}
\end{theorem}
\begin{proof}
$\ $\newline
\begin{itemize}
	\item[(1)] Let $I:S_1\rightarrow S_1$ be defined by $I(x)=x$. Then
		$I(I(x))=x$ so $I=I^{-1}$. Let $U_1\in \tau_1$, then
		$I^{-1}(U_1)=I(U_1)=U_1\in\tau_1$. Therefore $T_1\simeq T_1$.
	\item[(2)] The result follows trivially from
		Definition~\ref{homeomorphism}. Simply note that $g$ in the definition
		is also a homeomorphism.
	\item[(3)] Let $f:S_1\rightarrow S_2$ with $f_*$ like $g$ in
		Definition~\ref{homeomorphism}. Likewise for $g:S_2\rightarrow S_3$.
		Let $U\in\tau_3$, then $g^{-1}(U)\in\tau_2$ and thus $(f^{-1}\circ
		g^{-1})(U)\in\tau_1$. Next let $U'\in\tau_1$, then
		$f_*^{-1}(U')\in\tau_2$ and thus $(g_*^{-1}\circ
		f_*^{-1})(U)\in\tau_3$. So the functions $g\circ f$ and 
		$f_*\circ g_*$ are continues. Next note that 
		\begin{equation}
		\begin{aligned}
			((f_*\circ g_*)\circ (g\circ f))(x)&=x\\
			((g\circ f)\circ (f_*\circ g_*))(y)&=y
		\end{aligned}
		\end{equation}
		for $x\in S_1$ and $y\in S_3$ by hypothesis. Therefore $T_1\simeq T_3$.
\end{itemize}
\end{proof}
There are few other relevant definitions, they will follow in no particular
order.
\begin{definition}[Limit Point]
	Let $T=(S,\tau)$ be a topological space. Let $A\subset S$. A point $x\in S$
	is a \textit{limit point of} $A$ if and only if every open neighborhood
	$U\in\tau$ of $x$ satisfies:
	\begin{equation}
	\begin{aligned}
		A\cap (U\backslash \{x\})\neq\emptyset
	\end{aligned}
	\end{equation}
	That is there are atleast some points in the open neighborhoods
	of $x$ are in $A$.
\end{definition}
\begin{definition}[Derived Set]
	Let $T=(S,\tau)$ be a topological space. Let $X\subset S$. 
	The \textit{derived set} of $X$ is the set of all limit points of $X$,
	often denoted $X'$.
\end{definition}
\begin{definition}[Closure]
	Let $T=(S,\tau)$ be a topological space. Let $H\subset S$. 
	The \textit{closure of $H$ (in $T$)} is defined as:
	\begin{equation}
	\begin{aligned}
		H^-:=H\cup H'
	\end{aligned}
	\end{equation}
	Where $H$ is the derived set of $H$.
\end{definition}
\begin{definition}[Interior]
	Let $T=(S,\tau)$ be a topological space. Let $H\subset S$. 
	The \textit{interior} of $H$ is the union of all subsets of $H$ which are
	open in $T$. That is the interior $H^\circ$ of $H$ is defined as:
	\begin{equation}
	\begin{aligned}
		H^\circ:=\bigcup_{K\in\mathbb{K}} K
	\end{aligned}
	\end{equation}
	where $\mathbb{K}=\{K\in\tau:K\subseteq H\}$
\end{definition}
\begin{definition}[Boundary]
	Let $T=(S,\tau)$ be a topological space. Let $H\subset S$. 
	The \textit{boundary of} $H$ consists of all the points in the closure of
	$H$ that are not in the interior. Thus, the boundary of $H$ is defined as:
	\begin{equation}
	\begin{aligned}
		\del H := H^-\backslash H^\circ
	\end{aligned}
	\end{equation}
	where $H^-$ indicates the closure of $H$ and $H^\circ$ the interior of $H$.
\end{definition}
\newpage
\section{Homology}
Homology is about trying to understand topological spaces using group theory, to
this effect we will need the following definition.
\begin{definition}[Free abelian group]\label{Free abelian group}
Let $X\neq \emptyset$ such that $\left|X\right|=n\in \N$. Define
\begin{equation}
\begin{aligned}
	\Z X = \left\{\sum_{i=1}^n c_i\cdot x_i:c_i \in \Z \land x_i \in X\right\}
\end{aligned}
\end{equation}
The sum of two elements $a,b\in \Z X$ is defined as
\begin{equation}
\begin{aligned}
	a+b = \left(\sum_{i=1}^n a_i\cdot x_i\right)+
	      \left(\sum_{i=1}^n b_i\cdot x_i\right)
		= \left(\sum_{i=1}^n (a_i+b_i)\cdot x_i\right)
\end{aligned}
\end{equation}
Then we call $\Z X$ the \textit{free abelian group generated by} $X$.
\end{definition}
\begin{definition}[Abelian group axioms]\label{Abelian group}
Let $G$ be a set with an associated operation $+$. Then we call the ordered pair
$(G,+)$ an \textit{abelian group} if the following holds
	\begin{equation*}
	\begin{aligned}
		(G0)\; &:\quad \forall x,y\in G &&: x+y\in G 
		&&&\text{(Closure)}\\
		(G1)\; &:\quad \forall x,y,z\in G &&: x+(y+z)=(x+y)+z
		&&&\text{(Associativity)}\\
		(G3)\; &:\quad \exists 0\in G:\forall x\in G &&: 0+x=x+0=x
		&&&\text{(Identity)}\\
		(G4)\; &:\quad \forall x\in G:\exists -x\in G &&: x+(-x)=(-x)+x=0
		&&&\text{(Inverse)}\\
		(C)\; &:\quad \forall x,y\in G&&: x+y=y+x
		&&&\text{(Commutativity)}\\
	\end{aligned}
	\end{equation*}
\end{definition}
\begin{theorem}[The free abelian group is abelian group]
	The free abelian group (as defined in definition~\ref{Free abelian group})
	satisfy the axioms in definition~\ref{Abelian group}
\end{theorem}
\begin{proof}
	Clearly the axioms  are satisfied, since the operation on the free abelian
	group is defined in terms of ordinary addition.
\end{proof}
\begin{definition}[Homomorphism]\label{Homomorphism}
Let $(G,\circ)$ and $(H,*)$ be algebraic structures. We call a mapping
$\varphi:G\rightarrow H$ a \textit{homomorphism} if for any $g,h \in G$
\begin{equation}
\begin{aligned}
	\varphi(g\circ h)= \varphi(g)*\varphi(h)
\end{aligned}
\end{equation}
\end{definition}
\begin{theorem}[Homomorphism preserves abelian group structure]
	Assume that $(G,\circ)$ is an abelian group and that there exists an
	homomorphism $\varphi: G \rightarrow H$ where $(H,*)$ is an algebraic
	structure. Then $(H,*)$ is an abelian group.
\end{theorem}
\begin{proof}
	We will proves each axioms holds separately:
	\begin{enumerate}
		\item Clearly we have closure, since for all elements hit by $\varphi$. 
			Otherwise it would not be a map from $G \rightarrow H$
		\item Let $a,b,c \in G$ then we have
			\begin{equation}
			\begin{aligned}
				\varphi(a)*(\varphi(b)*\varphi(c)) 
											 &=\varphi(a)*\varphi(b\circ c)\\
											 &= \varphi(a\circ(b\circ c))\\
											 &= \varphi((a\circ b)\circ c) \\
											 &= \varphi(a\circ b)*\varphi(c)\\
											 &= (\varphi(a)*\varphi(b))*\varphi(c)
			\end{aligned}
			\end{equation}
		\item Let $e_G\in G$ be the identity element of $G$, then for $a\in G$
			\begin{equation}
			\begin{aligned}
				\varphi(a) = \varphi(a\circ e_G) = \varphi(a)*\varphi(e_G) \\
				\varphi(a) = \varphi(e_G\circ a) = \varphi(e_G)*\varphi(a) \\
			\end{aligned}
			\end{equation}
		\item Let $a^{-1} \in G$ be the inverse of $a\in G$, then
			\begin{equation}
			\begin{aligned}
				\varphi(e_G)=\varphi(a\circ a^{-1})=\varphi(a)*\varphi(a^{-1})\\
				\varphi(e_G)=\varphi(a^{-1}\circ a)=\varphi(a^{-1})*\varphi(a)\\
			\end{aligned}
			\end{equation}
		\item Let $a,b\in G$ then
			\begin{equation}
			\begin{aligned}
				\varphi(a)*\varphi(b) = \varphi(a\circ b) = \varphi(b\circ a)
				= \varphi(b)*\varphi(a)
			\end{aligned}
			\end{equation}
	\end{enumerate}
\end{proof}
\begin{definition}[Isomorphism]\label{Isomorphism}
Let $(G,\circ)$ and $(H,*)$ be algebraic structures and let
$\varphi:G\rightarrow H$ be a homomorphism. We call $\varphi$ an
\textit{isomorphism} if there exists an homomorphism $\psi : H\rightarrow G$
such that for all $g\in G$ and $h\in H$ we have
\begin{equation}
\begin{aligned}
	\varphi(\psi(h))=h\land \psi(\varphi(g))=g
\end{aligned}
\end{equation}
We say that $G$ and $H$ are isomorphic, writing $G\simeq H$.
\end{definition}
\begin{lemma}[Bijective homomorphism is isomorphism]\label{Bijective homomorphism is isomorphism}
	Let $\varphi:G\rightarrow H$ be a homomorphism from an algebraic structure
	$G$ to $H$. If $\varphi$ is bijective, then it is also a isomorphism.
\end{lemma}
\begin{proof}
	It's sufficient to show there exists a $\psi$ as in
	definition~\ref{Isomorphism}. We have by assumption that there is a
	homomorphism $\varphi:G\rightarrow H$ that is surjective and injective, that
	is
	\begin{equation}
	\begin{aligned}
		&\forall g,g'\in G: \varphi(g)=\varphi(g')\implies g=g'\\
		&\forall h\in H:\exists g\in G: \varphi(g)=h
	\end{aligned}
	\end{equation}
	Define $\psi:H\rightarrow G$ by $\psi(\varphi(g))=g$, this is possible since
	every $h\in H$ has a unique element in $g\in G$ such that $\varphi(g)=h$ by
	definition.
\end{proof}
\begin{lemma}[Surjective homomorphism onto $\Z$ iff hits
	identity]\label{surjective homomorphism condition}
	A homomorphism $\varphi:G\rightarrow \Z$ from a group $G$ is
	surjective iff there is some $g_1\in G$ such that $\varphi(g_1)=1\in\Z$
\end{lemma}
\begin{proof}
\textbf{Necessary condition}: Suppose that $\varphi$ is surjective, then by
definition there must be some $g_1\in G$ such that $\varphi(g_1)=1$.
\newline\newline
\textbf{Sufficient condition}: We will prove it by induction:

\textit{Base case}: By assumption, We have some $g_1\in G$ such that
$\varphi(g_1)=1$. Notice that since $\varphi$ preserves group structure there is
also some element of $g_{-1}\in G$ that maps to $-1$.

\textit{Inductive step}: Assume there is some $g\in G$ such that
$\varphi(g)=n\in \Z$. Then
\begin{equation}
\begin{aligned}
	n+1&=\varphi(g)+\varphi(g_1)=\varphi(g\circ g_1) \\
	n-1&=\varphi(g)+\varphi(g_{-1})=\varphi(g\circ g_{-1})
\end{aligned}
\end{equation}
By the closure property of groups this completes the proof.
\end{proof}
\begin{definition}[Singular $p$-simplex]\label{singular p-simplex}
	Let $p\in\{0,1,2,3\}$ and let $\Delta^p$ be the topological space
	\begin{equation}
	\begin{aligned}
		\Delta^p = \left\{ (t_0,\dots,t_p)\in\R^{p+1}:\sum_{i=0}^p t_i= 1\land
		0\leq t_i\right\}
	\end{aligned}
	\end{equation}
\end{definition}
\begin{definition}[Wall-map]\label{Wall-map}
	For $0\leq i\leq p$, for $p\in\{0,1,2,3\}$, let\newline 
	$\delta^i: \Delta^{p-1}\rightarrow\Delta^p$ be the function defined by
	\begin{equation}
	\begin{aligned}
		\delta^i(t_0,\dots,t_{p-1}) = (t_0,\dots,t_{i-1},0,t_{i},\dots,t_{p-1})
	\end{aligned}
	\end{equation}
	Further more we set $\Delta^{-1}=\emptyset$. We call these maps
	\textit{wall-maps}.
\end{definition}
\begin{definition}[The singular $p$-simplexs of $X$]\label{singular p-simplex x}
	Let $X$ be a topological space, let $p \in \{0,1,2,3\}$ and define the set
	\begin{equation}
	\begin{aligned}
		\Sin_p(X)=\{\sigma:\Delta^p\rightarrow X: \sigma \in C(\sigma^p,X)\}
	\end{aligned}
	\end{equation}
	We call $\Sin_p(X)$ \textit{the singular $p$-simplexs of $X$}.
\end{definition}
\begin{definition}\label{di wall-map}
	Let $X$ be a topological manifold. Let $0\leq i\leq p$, for\newline
	$p\in\{0,1,2,3,4\}$, let $d_i:\Sin_p(X)\rightarrow \Sin_{p-1}(X)$ be given
	by
	\begin{equation}
	\begin{aligned}
		d_i(\sigma)=\sigma \circ\delta^i
	\end{aligned}
	\end{equation}
\end{definition}
\begin{theorem}
	Let $X$ be a topological space. Let $\sigma\in \Sin_p(X)$ and $0\leq i<j\leq
	p$. Then $d_i d_j(\sigma)=d_{j-1}d_i(\sigma)$.
\end{theorem}
\begin{proof}
	First note that
	\begin{equation}
	\begin{aligned}
		d_i d_j(\sigma) &=\sigma\circ\delta^j\circ\delta^i\\
		d_{j-1}d_i(\sigma) &=\sigma\circ\delta^i\circ\delta^{j-1}
	\end{aligned}
	\end{equation}
	For the first we write (by definition~\ref{singular p-simplex})
	\begin{equation}
	\begin{aligned}
		(\sigma\circ\delta^j\circ\delta^i)(t_0,\dots,t_p) &=
		(\sigma\circ\delta^j)(t_0,\dots,t_{i-1},0,t_{i},\dots,t_{p-1})\\
		&= \sigma(t_0,\dots,t_{i-1},0,t_i,\dots,t_{j-1},0,t_j\dotsi,t_{p-1})
	\end{aligned}
	\end{equation}
	For the second
	\begin{equation}
	\begin{aligned}
		(\sigma\circ\delta^i\circ\delta^{j-1})(t_0,\dots,t_p) &=
		(\sigma\circ\delta^i)(t_0,\dots,t_{j-2},0,t_{j-1},\dots,t_{p-1})\\
		&= \sigma(t_0,\dots,t_{i-1},0,t_i,\dots,t_{j-1},0,t_j\dotsi,t_{p-1})
	\end{aligned}
	\end{equation}
	Since the element was arbitrary, this completes the proof.
\end{proof}
\begin{definition}[Singular $p$-chain in $X$]\label{singular p-chain}
	Let $X$ be a topological room. For $p\in\{0,1,2,3\}$, define
	\begin{equation}
	\begin{aligned}
		C_p(X)=\Z\Sin_p(X)=\left\{\sum_{\sigma\in\Sin_p(X)}c_\sigma\cdot\sigma:
		c_\sigma\in\Z:|\{c_\sigma:c_\sigma\neq0\}|\in\N\right\}
	\end{aligned}
	\end{equation}
	that is the free abelian group generated by $\Sin_p(X)$. We call this
	abelian group the \textit{singular $p$-chains in $X$}.
\end{definition}
\begin{definition}[$p^\text{th}$ chain map]\label{pth chain map}
	Let $\del_p:C_p(X)\rightarrow C_{p-1}(X)$, for $p\in\{0,1,2,3\}$ be the map
	given by
	\begin{equation}
	\begin{aligned}
		\del_p\sigma=\sum^p_{i=0} {(-1)}^i d_i (\sigma)
	\end{aligned}
	\end{equation}
	where $\sigma\in\Sin_p(X)$. We call this map the
	\textit{$p^\text{th}$ chain map}. For $s\in C_p(X)$ the function the map is
	given by
	\begin{equation}
	\begin{aligned}
		\del_ps=\sum_{\sigma\in\Sin_p(X)}\left(\sum^p_{i=0} {(-1)}^i c_\sigma 
			d_i(\sigma)\right)
	\end{aligned}
	\end{equation}
\end{definition}
\begin{theorem}
	Let $\sigma\in\Sin_p(X)$, then
	\begin{equation}
	\begin{aligned}
		\del_{p-1}\del_p(\sigma)=0
	\end{aligned}
	\end{equation}
\end{theorem}
\begin{proof}
\end{proof}
\begin{definition}[Subgroup]\label{subgroup}
	Let $(G,*)$ be an abelian group. Let $H\subseteq G$, then $(H,*)$
	is called a \textit{subgroup} of $(G,*)$ iff $(H,*)$ satisfies
	the axioms in definition~\ref{Abelian group}. This we write as
	$H\trianglelefteqG$. If $H=\{e_G\}$, that is $H$ contains only the
	identity of $G$ we call this the \textit{trivial subgroup} of $G$.
	We call $H$ a \textit{proper subgroup} if $H\subset G$.
\end{definition}
\begin{definition}[Coset]\label{coset}
	Let $G$ be a abelian group and let $H$ be a subgroup of $G$. We define for
	$g\in G$
	\begin{equation}
	\begin{aligned}
		g+H &=\left\{g*h:h\in H\right\}\\
		H+g &=\left\{h*g:h\in H\right\}
	\end{aligned}
	\end{equation}
	We call these sets the \textit{cosets} of $H$.
\end{definition}
\begin{theorem}
	Let $H$ be a subgroup of $G$. Let $g\in G$. Then
	\begin{equation}
	\begin{aligned}
		g+H &=H+g
	\end{aligned}
	\end{equation}
\end{theorem}
\begin{proof}
	Since $G$ is abelian, $H$ must be too. Therefore $g*h=h*g$ for $g\in G$ and
	$h\in H$ which completes the proof.
\end{proof}
\begin{theorem}
	Let $H$ be a subgroup of $G$. Let $g,g'\in G$. Then
	\begin{equation}
	\begin{aligned}
		g'+(g+H) &= (g'*g)+H
	\end{aligned}
	\end{equation}
\end{theorem}
\begin{proof}
	First note	
	\begin{equation}
	\begin{aligned}
		g'+(g+H) &= g'+\{g*h:h\in H\}\\
				 &= \{g'*(g*h):h\in H\}
	\end{aligned}
	\end{equation}
	Since $G$ is an abelian group, we have
	\begin{equation}
	\begin{aligned}
		\{g'*(g*h):h\in H\} &= \{(g'*g)*h:h\in H\}
	\end{aligned}
	\end{equation}
	But that is just $(g'*g)+H$.
\end{proof}
\begin{definition}[Product of group subsets]\label{group product}
	Let $S,T\subset G$ where $(G,*)$ is an abelian group. Then we define the
	product
	\begin{equation}
	\begin{aligned}
		ST = \left\{s*t:s\in S\land t\in T\right\}
	\end{aligned}
	\end{equation}
\end{definition}
\begin{theorem}
	Let $H$ be a subgroup of $G$. Let $g,g'\in G$. Then
	\begin{equation}
	\begin{aligned}
		(g+H)(g'+H) = (g*g')+H
	\end{aligned}
	\end{equation}
\end{theorem}
\begin{proof}
	First we prove that $(g*g')+H\subseteq (g+H)(g'+H)$. Note that
	\begin{equation}
	\begin{aligned}
		(g+H)(g'+H) &= \{a*b:a\in g+H\land b\in g'+H\}\\
					&= \{(g*h)*(g'*h'):g,g'\in G\land h,h'\in H\}\\
					&= \{(g*g')*(h*h'):g,g'\in G\land h,h'\in H\}
	\end{aligned}
	\end{equation}
	The last step are justified since $(G,*)$ is an abelian group. Note that
	since $(H,*)$ satisfies definition~\ref{Abelian group} we have $e_G\in H$,
	therefore be setting $h=e_G$ we obtain that $(g*g')+H\subseteq (g+H)(g'+H)$.
	Next we prove that $(g+H)(g'+H)\subseteq (g*g')+H$. 
	\begin{equation}
	\begin{aligned}
		(g*g')+H &= \{(g*g')*h:h\in H\}
	\end{aligned}
	\end{equation}
	We note that $h*h'\in H$ since $(H,*)$ has closure. Therefore we conclude
	that $(g+H)(g'+H)\subseteq (g*g')+H$ and thus $(g+H)(g'+H)=(g*g')+H$.
\end{proof}
\end{document}
