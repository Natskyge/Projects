\documentclass{article}
\usepackage[utf8]{inputenc}
\usepackage{amsmath}
\usepackage{amssymb}
\usepackage{amsthm}
\usepackage{amsfonts}
\newtheorem{theorem}{Theorem}[section]
\newtheorem{corollary}{Corollary}[theorem]
\newtheorem{lemma}{theorem}[Lemma]

\title{Definition of logical quantifiers}
\author{Erik. S. Gimsing}
\date{November 13, 2017}

\begin{document}

\maketitle

Let $P: A \rightarrow \{\top,\bot \}$, where $\neg \top = \bot$ and 
$\neg \bot = \top$

\begin{align}
	\forall x \in A,\ P(x) = \top 
	&\Leftrightarrow \bot \notin \{P(x)|x \in A\} \\
	\exists x \in A,\ P(x) = \top
	&\Leftrightarrow \top \in \{P(x)|x \in A\} \\
	\exists! x \in A,\ P(x) = \top &\Leftrightarrow \forall y \in A,\ 
	P(x) = P(y) = \top \Rightarrow x = y
\end{align}

From this it is trivial to prove the following:

\begin{align}
	\neg (\forall x \in A, P(x) = \top) &\Leftrightarrow 
	\neg (\bot \notin \{P(x)|x \in A\}) \\
	&\Leftrightarrow \bot \in \{P(x)|x \in A \} \nonumber \\
	&\Leftrightarrow \exists x \in A, P(x) = \bot \nonumber \\
	&\Leftrightarrow \exists x \in A, \neg P(x) = \top \nonumber
\end{align}

\begin{align}
	\neg (\exists x \in A, P(x)) &\Leftrightarrow 
	\neg ( \top \in \{P(x)|x \in A\} ) \\
	&\Leftrightarrow \top \notin  \{P(x)|x \in A\} \nonumber \\
	&\Leftrightarrow \forall x \in A, P(x) = \bot \nonumber \\
	&\Leftrightarrow \forall x \in A, \neg P(x) = \top \nonumber
\end{align}


\begin{align}
	p \leftrightarrow q \eq = (p \rightarrow q) \land (q \rightarrow p) \\
	p \rightarrow q = \neg p \lor q
\end{align}

\newpage

\section{Rectangle center of mass}
\textbf{Theorem}:
Consider a $n \times m$ rectangle. It is to be proven that the center of 
mass is $(n/2,m/2)$.

\textit{proof}:
Consider the points $A = (0,0),\ B = (n,0),\ C = (n,m)$ and $D = (0,m)$. Then 
$ABCD$ is a rectangle with sides $n$ and $m$. Thus the center of mass of it
is the same as the $n \times m$ rectangle. 

Let a circle have center $(n/2,m/2)$ and radius $r$. Given a point 
$p=(a,b)$ lying on the perimeter of the circle, then we may find another point 
on the other side of the circle, namely $p'=(n-a,m-a)$. If we then consider only
the points $p$ such that $p$ is inside the square, that is $0 \leq a \leq n$ 
and $0 \leq b \leq m$ then we find that $p'$ is in the rectangle and the center
of $p$ and $p'$ to be:

\begin{align}
	\left( \frac{a+(n-a)}{2},\frac{b+(m-b)}{2} \right) =
	\left( \frac{n}{2},\frac{m}{2} \right) \nonumber
\end{align}

Since $r$ is free to change then the points $p$ and $p'$ together are all points 
in the rectangle it follows that the center of mass is $(n/2,m/2)$.

\section{Symmetry}
\textit{Note: Equal distance}

A shape $X \subseteq \mathbb{R}^n$ is symmetric if there is $x \in \mathbb{R}^n$ 
such that for $S^n$ with center $x$ then for all $p \in S^n \cap X$ there
is a $P \subseteq S^n \cap X$ and 

\begin{align}
	(|P|+1)x = p + \sum_{q \in P} q \nonumber
\end{align}

\begin{align}
	\exists x \in \mathbb{R}^n:\ \forall p \in S^n \cap X\ \exists P \subseteq 
	S^n \cap X\ (|P|+1)x = p + \sum_{q \in P} q \nonumber
\end{align}

\section{Average}
Let $F$ be a field, then for $B \subseteq F$ the
\textit{average}, $a$, of $B$ is

\begin{align}
	a = \frac{1}{|B|}\sum_{b \in B} b
\end{align}

\newpage

\section{Symmetry 2}
A symmetry on a set $X$ is a bijective function $s: X \rightarrow X$ such that
for any operator $*$ on $X$, $s(x * y) = s(x) * s(y)$ and the
symmetries of $X$ is a set $\mathcal{S}$ containing all such functions $s$. 
We call the function $s(x)=x$ a \textit{trivial symmetry}, since it always 
exists and denote it $I$. Further more, since $s$ is bijective there is an $s^{-1}$ 
for all $s$. We say that that $X$ is has symmetry order of $n$ if there exists
$s_1,s_2,\dots,s_n$ such that $s_1 \circ s_2 \circ \cdots \circ s_n = I$. 

\section{Symmetry 3}
A symmetry is a bijective function on a mathematical structure.

\section{Condition for defining the natural numbers}
$\mathbb{N} \neq \emptyset$ and there must for all $a \in \mathbb{N}$ be a $b \in 
\mathbb{N}$ such that given an order $\prec$ on $\mathbb{N}$ then $a \prec b$.

\section{Various graph theory definitions}
\subsection{Graph}
\textbf{Def}: A graph $G$ is $(V,E)$ where in $V$ is a set whose elements we
call \textit{vertices's} and $E \subseteq \{\{x,y\}| x,y \in V\} = \binom{V}{2}$ 
whose elements we call an \textit{edge} and for any $\{x,y\} \in E$ 
(written $xy \in E$) we say there is an edge between $x$ and $y$. 

\subsection{Graph with movement}
\textbf{Def}: A graph $G=(V,E,M)$ where $V$ and $E$ are defined as above and 
$M = \{(x,y)| x,y \in z \in E\}$. That is for all $xy \in E$, $xy, yx \in M$.

We define $i_1(m) = m^1$ and $i_2(m) = m^2$, where $m = (a,b)$, such that 
$(i_1(m),i_2(m)) = (m^1,m^2) = m$.

Further more we can apply a movement on a vertice follows, let $v \in V$ and
$m \in M$:

\begin{equation}
	(|): V \times M \rightarrow V 
\end{equation}
\begin{equation}
	(v|m) = m^2 \Leftrightarrow v = m^1 \nonumber
\end{equation}

We can compose movements as follows using the operator $+$, let $n,m \in M$

\begin{align}
	n+m = n^1m^2 \Leftrightarrow n^2 = m^1
\end{align}

\subsection{Simple graph}
\textbf{Def}: A graph containing no loops, $\nexists x \in E,\ x = \{y,y\}$, or 
double edges, $E \subseteq \binom{V}{2}$.

\subsection{Complete graph}
\textbf{Def}: A complete graph is a graph such that $E = \binom{V}{2}$.

\subsection{Connected graph}
\textbf{Def}: For any $x,y \in V$ we call them connected if $xy \in E$ or there
exists $xz \in E$ such that $z$ and $y$ are connected. More precisely

\begin{equation}
	C: V \times V \rightarrow \{\top,\bot\}
\end{equation}
\begin{equation}
	C(x,y) = xy \in V \lor (\exists z \in V,\ xz \in E \land C(z,y)) \nonumber
\end{equation}

Then a graph $G$ is connected if for all $x,y \in V$ then $x$ and $y$ are
connected.

\subsection{Cyclic graph}
\textbf{Def}: A simple and connected graph where any $v\in V$ is in precisely 2
elements of $E$.
\begin{align*}
	\forall v \in V \exists e,f\in E: v\in e,f \land (v \in g \in E \implies (g
	=e \lor g = f)) \land e \neq f
\end{align*}

\section{Logarithm}
\begin{theorem}
	\begin{align*}
		\ln x = \int^x_1 \frac{dt}{t}
	\end{align*}
\end{theorem}
\begin{proof}
	First note that $e^x$ is bijective. So in fact there exists a function $\ln$
	such that $\ln e^x = e^{\ln x} = x$ so we may write
	\begin{align*}
		e^{\ln x} = x \\
		D_x e^{\ln x} = D_x x \\
		x D_x \ln x = 1 \\
		D_x \ln x = \frac{1}{x}
	\end{align*}
	We have that $e^0 = 1$, so $\ln 1 = 0$, therefore
	\begin{align*}
		D_x \ln x = \frac{1}{x} \\
		\ln x = \int^x_1 \frac{dt}{t}
	\end{align*}
\end{proof}
\newpage
\section{Teknologi noter}
\begin{enumerate}
	\item Ulemper med gas lys
	\begin{enumerate}
		\item Bruger ilt
		\item Brandefare
		\item Dyre
	\end{enumerate}
	\item Ulemper med kul lys
	\begin{enumerate}
		\item Meget dyrt
		\item Holder ikke lang tid
		\item For skarpt lys
	\end{enumerate}
\end{enumerate}

\section{List data structure}
An \textit{list data strucutre} is an order pair $\mathcal{L} = (L,E)$ where 
$L$ is a set of lists and $E$ a set of elements. The following operations are
defined on $\mathcal{L}$
\begin{align*}
	cons: E \times L \rightarrow L \\
	\text
\end{align*}

\end{document}
